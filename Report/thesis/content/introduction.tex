%!TEX root = ../thesis.tex
\section{Motivation}

\paragraph{Detlev} Rich text ist sauwichtig (geschichte, lulturgut). So was wie google docs ist im moment NICHT MÖGLICH. Wegen contenteditable. Ich ändere das.

AUSSERDEM KEIN EDITOR SONDER EINE LIBRARY ganz tolles feature und wichtiger Punkt


Rich-text editors are commonly used by many on a daily basis. Often, this happens knowingly, for instance in an office suite, when users wilfully format text. But often, rich-text editors are being used without notice. For instance when writing e-mails, entering a URL inserts a link automatically in many popular e-mail-applications. Also, many applications, like note-taking apps, offer rich-text capabilites that go unnoticed. Many users do not know the difference between rich-text and plain-text writing. Rich-text editing has become a de-facto standard, that to many users is \textit{just there}. Even many developers do not realise that formatting text is a feature that needs special implementation, much more complex than plain-text editing.

While there are APIs for creating rich-text input controls in many desktop programming environments, web-browsers do not offer native rich-text inputs. However, third-party JavaScript libraries fill the gap and enable developers to include rich-text editors in web-based projects.

The libraries available still have downsides. Most importantly, only a few of them work. As a web-developer, the best choices are either to use CKEditor or TinyMCE. Most other editors are prone to bugs and unwanted behaviour. Piotrek Koszuli\'{n}ski, core developer of CKEditor comments this on StackOverflow as follows:

\begin{quotation}
\textit{''Don't write wysiwyg editor[sic] - use one that exists. It's going to consume all your time and still your editor will be buggy. We and guys from other... two main editors (guess why only three exist) are working on this for years and we still have full bugs lists ;).\cite{sopp}''}
\end{quotation} %\footnote{\url{http://stackoverflow.com/questions/11240602/paste-as-plain-text-contenteditable-div-textarea-word-excel/11290082#11290082}, last checked on 07/13/2015}

A lot of the bugs CKEditor and other editors are facing are due to the fact that they rely on so-called ''HTML Editing APIs'' that have been implemented in browsers for years, but only been standardized with HTML5. Still, to this present day, the implementations are prone to noumerous bugs and behave inconsitently across different browsers. And even though these APIs are the de-facto standard for implementing rich-text editing, with their introduction in Internet Explorer 5.5, it has never been stated they have been created to be used as such.

It's a fact, that especially on older browsers, rich-text editors have to cope with bugs and inconsistencies, that can only be worked around, but not fixed, as they are native to the browser. On the upside, these APIs offer a high-level API to call so-called ''commands'' to format the current text-selection. 

However, calling commands will only manipulate the document's DOM tree, in order to format the text. This can also be achieved without using editing APIs, effectively avoiding unfixable bugs and enabling a consistend behaviour across all browsers.

Furthermore CKEditor, TinyMCE and most other libraries are shipped as user interface components. While being customizable, they tend to be invasive to web-projects.

This thesis demonstrates a way to enable rich-text editing in the browser without requiring HTML Editing APIs, provided as a GUI-less software library. This enables web-developers to implement rich-text editors specific to the requirements of their web-projects.

\textit{Rich-text editing on the web is a particularly overlooked topic. Most libraries use contenteditable without questioning its benefits. The literature on this topic is thin. It is rarely written about in books and papers and noone really examines alternative ways for implementation. ACE and CodeMirror show techniques how to do it. However looking at its history, it seems very questionable. People who implement editors using it often rant about its disadvantages. <- der letzte Satz sollte einer der kernpunkte der introduction sein. Überahupt, das questioning sollte im kern stehen. SCHREIBEN DASS AUCH ANDERE WIE MEDIUM DARÜBER RAGEN, SOWIESO MEHR REFERENZEN FINDEN}

\paragraph{detlev} Warum textverarbeitung im Web? - 1 applikation auf allen systemem anstatt 1000 programme und versionen, weniger wartbar aufwändiger - weltweit verfügbar - gebackupped durch hosting und cloud - und wirtschaftlich zentral verwetbar. - probleme der pirvatsphäre sind nicht teile dieser arbeit.

\section{Terminilogy}

In the web-development world, the term \textit{WYSIWYG} editor is commonly used. WYSIWYG is an abbrevation for \textbf{W}hat \textbf{Y}ou \textbf{S}ee \textbf{I}s \textbf{W}hat \textbf{Y}ou \textbf{G}et and descibes a text editor's capability to display formatted text as it is being edited. This stands out to plain-text editors that can neither display nor edit formattings. The term rich-text editor has often been used for this feature and is more precise. For this reason, the term \textit{rich-text editor} and \textit{rich-text editing} will be used in this thesis.

%rich-text, WYSIWYG, word-processing, WYSIWYM

\section{Structure}

The first part of this thesis explains rich-text editing on desktop PCs. The second part explains how rich-text editors are currently being implemented in a browser-environment and the major technical differences to the desktop. Part three will cover the downsides and the problems that arise with the current techniques used. Part four will explain how rich-text editing can be implemented on the web bypassing these problems. Part five dives into the possibilities of web-based rich-text editing in particular when using the techniques explained in this thesis.

%% possibilites
%% * collaboration with etherpad
%% * Markdown

% desktop -> caret selection etc -> contenteditable not necessary ->
% we will see that contenteditable handles all this, but in my solution this all needs to be coded again
% I need to discuss that the benefits are bigger than having to do this