%!TEX root = ../../thesis.tex


\section{Mobile Support}

It's technically there, it must be tested properly


\subsection{BiDi \& IME support}

http://marijnhaverbeke.nl/blog/cursor-in-bidi-text.html

http://marijnhaverbeke.nl/blog/browser-input-reading.html 

https://en.wikipedia.org/wiki/Input\_method


\section{Development / Meta}
Crockford style is a bad idea. 
I will change it to Standard or Airbnb 
\url{https://github.com/airbnb/javascript/tree/master/es5}


\section{Outlook}

Over time, the bugs of HTML editing APIs will decrease. Its clipboard capabilities are on the way to be expanded. The API still is still limited and needs a revision. It is even imaginable to rethink the way \code{contenteditable} works. Editors that, for instance, implement layouting, like Google's document editor, still cannot be implemented with the way HTML Editing APIs are designed.

To allow a transition from current HTML editing APIs and an interface with a cleaner and richer functionality, it is thinkable to introduce a new ''class'' alongside the old API. This has been done with other functionality, for example mal aus MDN raussuchen. This way the old API can die gracefully while web developers slowly adopt. It can be hoped that if the API is much better, the adoption will happen quickly.


As discussed in \refsubsec{subsec:edit_api_adv_thir_party_lang} my design as a library with a super duper api allows implementing highlighting for other languages like bb code or markdown. \textit{There should be a part in CONCEPT that explains this idea, either explaining its made for extensibility or in how cool my api is i mean the design as a lib and not as an editor is}


besseres undo durch erkennen von ganzen worten (wenn man leerzeichen und so drückt)

Events zu allen gelegenheuten triggern für andere developer


DOCH document model benutzen weil der shit von prosemirror einfach so geil ist.

Auf der anderen Seite ist so was wie der Medium editor mit meiner Version viel besser