%!TEX root = ../../thesis.tex

\chapter{Evaluation}
\label{ch:evaluation}

The library of this thesis implements basic features or rich-text editing. Users can type inside an editor, format text, place and move a caret, create a selection, use undo/redo commands and copy and paste text. Some of the intended features  have not been implemented. These features include:

\begin{enumerate}
\item Behavior of the enter key in lists and headlines.
\item Contents pasted from the clipboard are not being sanitized.
\item Lists cannot be edited.
\item Formatting does not allow for style and class attributes and do not distinguish between inline and block commands.
\item Real-time collaboration is unstable.
\end{enumerate}

The basic system for all of these features could be implemented. These features are missing due to the many small tasks they require. On their own, these tasks are not very complex. 

\begin{enumerate}
\item For the behavior of the enter key, input filters must be implemented that check which element the caret is contained by and create either new list items or text paragraphs.
\item To sanitize paste events, an input filter must be implemented that removes tags from pasted contents.
\item Lists can already be inserted and edited, but they cannot be removed. The input filter that deletes text must also account for lists.
\item To allow formatting with style and class attributes, the tags with these attributes must be distinguished from tags with the same tag name, but different attributes.
\item Block and inline formattings behave differently when being applied. To decide which type of formatting to apply, the tag name of the formatting command must be checked and a distinct algorithm for block formattings must be implemented.
\item Real-time collaboration has many edge cases which need to be accounted for and bugs need to be fixed.
\end{enumerate}



Web developers can implement editors
and extend
etherpad shows such an extension


\chapter{Outlook}

Can be done after thesis
Input module for lists




Google's document editor has shown that it is possible to implement a rich text editor without editing APIs
The resulting library of this thesis demonstrates a way to implement a rich-text editor on the web without using HTML editing APIs.
The library includes most basic features of a rich-text editor
users can write, delete, format, control a caret, select text, copy \& paste
the library provides a system for extension
users can already use the advantage of the web and perform real-time collaboration
The editor however only implements a part of the functionality of rich text editors
input behavior (enter in lists and headlines)

Outlook

--Detlev--

Habe ich es erreicht im Sinne der Kulturtechnik es leichter zu machen text Editoren zu erstellen (bugfrei?).
Habe ich es geschafft?
 - Grundelemente von rich-text editor hergestellt? (caret etc)
 - Kollaboration? (die sogar jeder andere verwenden kann?)
 - kann es jedem zugänglich gemacht werden? (jeder kann es als JS library einsetzten)
 - Erweiterbar? zB Wörterbücher
 - Kann es auf allen Browsern und OSs verwendet werden?
 - Kann es mobil verwendet werden?




\section{Features}

The resulting library of this thesis demonstrates a way to implement a rich-text editor on the web without using HTML editing APIs. With Type, web developers can turn any elements into an editable sections providing essential functionality for the user to write, remove and format the text. Developers using the library are provided with an API to implement an own editor and are able to control the caret, selection and undo/redo functions above editing the text contents. Using the Etherpad extension, developers can connect to an Etherpad server to allow for real-time collaboration in their own editor. The library provides an ecosystem for extensions that allows developers to add functionality.

%The library can be extended and improved by third-party developers. Type

%It can be integrated in any website to allow other developers


%The resulting editor of this thesis is capable of basic rich-text editing including real-time collaboration. 

%\begin{itemize}
%\item Text can be written and removed.
%\item Text can be formatted.
%\item The caret can be controlled.
%\item The text selection can be read and controlled.
%\item An editor can be connected to an Etherpad server for real-time collaboration alongside Etherpad's own clients.
%\item Changes can be undone and redone accounting for multiple input sources (like Etherpad).
%\item An editor can be interacted with like a native input field.
%\end{itemize}


 
 
\section{MVC \& Document model}

Towards the end of this thesis, Marijn Haverbeke, author of CodeMirror, started a crowd-funding campaign for supporting him to build a rich-text editor that does not rely on HTML editing APIs---called ProseMirror---just like library of this thesis. As discussed in \refsection{sec:mvc_architecture}, Type does not rely on an architecture that abstracts the contents of the editor from the DOM. ProseMirror, by contrast, takes this approach. This restricts working with the editor to the capabilities of the internal model of the document and might make extending the editor complicated. On the other hand, it allows for a very stable rendering of the contents and (as discussed) switching between HTML or Markdown output. It is good to see both approaches being realized. By contrasting both approaches in a practical manner it might be easier to decide which approach is better for which purposes.

%\chapter{Outlook}
%\label{ch:outlook}

\section{Outlook}

Notiz: Block formatting fehlt

Type's features are not complete yet. Common behavior for rich-text editing as well as input method (IME) support must be implemented though input filters. The caret needs bi-directional text support. Type requires thorough testing across browsers and mobile devices. To support developers Type can trigger events, for instance when the caret moves or the selection changes.

Editing images and tables is, to the biggest part, a matter of implementing a user interface, but Type can support the development with utility methods. This can possibly provided as extensions, distributed as own files.


%Over time, the bugs of HTML editing APIs will decrease. Its clipboard capabilities are on the way to be expanded. The API still is still limited and needs a revision. It is even imaginable to rethink the way \code{contenteditable} works. Editors that, for instance, implement layouting, like Google's document editor, still cannot be implemented with the way HTML Editing APIs are designed.

%To allow a transition from current HTML editing APIs and an interface with a cleaner and richer functionality, it is thinkable to introduce a new ''class'' alongside the old API. This has been done with other functionality, for example mal aus MDN raussuchen. This way the old API can die gracefully while web developers slowly adopt. It can be hoped that if the API is much better, the adoption will happen quickly.


%As discussed in \refsubsec{subsec:edit_api_adv_thir_party_lang} my design as a library with a super duper api allows implementing highlighting for other languages like bb code or markdown. \textit{There should be a part in CONCEPT that explains this idea, either explaining its made for extensibility or in how cool my api is i mean the design as a lib and not as an editor is}


%besseres undo durch erkennen von ganzen worten (wenn man leerzeichen und so drückt)

%Events zu allen gelegenheuten triggern für andere developer


%DOCH document model benutzen weil der shit von prosemirror einfach so geil ist.

%Auf der anderen Seite ist so was wie der Medium editor mit meiner Version viel besser

%\section{Evaluation notizen}

%There are some things that need to be done
% - Bugs, tests
% - Mobile support
% - BiDi \& IME support
% - A more profound model to sync with etherpad
% - trigger more events
% - images, tables, lists





%\section{Mobile Support}

%It is technically there, it must be tested properly


%\subsection{BiDi \& IME support}

%http://marijnhaverbeke.nl/blog/cursor-in-bidi-text.html

%http://marijnhaverbeke.nl/blog/browser-input-reading.html 

%https://en.wikipedia.org/wiki/Input\_method


%\section{Development / Meta}
%Crockford style is a bad idea. 
%I will change it to Standard or Airbnb 
%\url{https://github.com/airbnb/javascript/tree/master/es5}


%\section{Outlook}

