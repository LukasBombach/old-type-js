%!TEX root = ../../thesis.tex
\section{Emergence of HTML editing JavaScript libraries}

Around the year 2003\footnote{compare \textit{Meine Tabelle aller Editoren}} the first JavaScript libraries emerged that made use of Microsoft's and Mozilla's editing mode to offer rich-text editing in the browser. Usually these libraries were released as user interface components (text fields) with inherent rich-text functionality and were only partly customizable.

In May 2003 and March 2004 versions 1.0 of ''FCKEditor''\footnote{Now distributed as ''CKEditor''} and ''TinyMCE'' have been released as open source projects. These projects are still being maintained and remain among the most used rich-text editors. TinyMCE is the default editor for Wordpress and CKEditor is listed as the most popular rich-text editor for Drupal\footnote{\url{https://www.drupal.org/project/project\_module}, last checked on 07/16/2015}. 

Since the introduction of Microsoft's HTML editing APIs, a large number of rich-text editors have been implemented. While many have been abandoned, GitHub lists about 600 JavaScript projects related to rich-text editing\footnote{\url{https://github.com/search?o=desc\&q=wysiwyg\&s=stars\&type=Repositories\&utf8=\%E2\%9C\%93}, last checked on 07/16/2015}. However, it should be noted, that some projects only use other projects' editors and some projects are stubs. Popular choices on GitHub include ''MediumEditor'', ''wysihtml'', ''Summernote'' and others.
