%!TEX root = ../../thesis.tex
\section{HTML Editing APIs}
\label{sec:html-editing-apis}

In July 2000, with the release of Internet Explorer 5.5, Microsoft introduced the IDL attributes \texttt{contentEditable} and \texttt{designMode} along with the content attribute \texttt{contenteditable}\footnote{\url{https://msdn.microsoft.com/en-us/library/ms533720(v=vs.85).aspx}, last checked on 07/10/2015}\footnote{\url{https://msdn.microsoft.com/en-us/library/ms537837(VS.85).aspx}, last checked on 07/10/2015}. These attributes were not part of the W3C's HTML 4.01 specifications\footnote{\url{http://www.w3.org/TR/html401/}, last checked on 07/14/2015} or the ISO/IEC 15445:2000\footnote{\url{http://www.iso.org/iso/iso\_catalogue/catalogue\_tc/catalogue\_detail.htm?csnumber=27688}, last checked on 07/14/2015}, the defining standards of that time. Table \ref{table:editing-api-attributes} lists these attributes and possible values.

% Please add the following required packages to your document preamble:
% \usepackage{graphicx}
\begin{table}[]
\centering
\resizebox{\textwidth}{!}{%
\begin{tabular}{llll}
\hline
Attribute       & Type              & Can be set to         & Possible values                     \\ \hline
designMode      & IDL attribute     & Document              & "on", "off"                         \\
contentEditable & IDL attribute     & Specific HTMLElements & boolean, "true", "false", "inherit" \\
contenteditable & content attribute & Specific HTMLElements & empty string, "true", "false"       \\ \hline
\end{tabular}
}
\caption{Editing API attributes}
\label{table:editing-api-attributes}
\end{table}

\begin{lstlisting}[language=html, caption=An element set to editing mode, label=lst:div-contenteditable]
<div contenteditable="true">
  This text can be edited by the user.
</div>
\end{lstlisting}

By setting \texttt{contenteditable} or \texttt{contentEditable} to ''true'' or \texttt{designMode} to ''on'', Internet Explorer 5.5 switches the affected elements and their children to an editing mode. The \texttt{designMode} can only be applied to the entire document and the \texttt{contentEditable} and \texttt{contenteditable} attributes can be applied to specific HTML elements as described on Microsoft's Developer Network (MSDN) online documentation\footnote{\url{https://msdn.microsoft.com/en-us/library/ms537837(VS.85).aspx}, last checked on 07/10/2015}. These elements include ''divs'', ''paragraphs'' and the docuement's ''body'' element amongst others. In editing mode

\begin{enumerate} \item Users can interactively click on and type inside texts \item An API is enabled that can be accessed via JScript and JavaScript\end{enumerate}

Setting the caret by clicking on elements, accepting keyboard input and modifying text nodes is handled entirely by the browser. No further scripting is necessary.

The API enabled can be called globally on the \texttt{document} object, but will only execute when the user's selection or caret is focussed inside an element in editing mode. Table \ref{table:editing-mode-api} lists the full HTML editing API. To format text, the method \texttt{document.execCommand} can be used. Calling

\begin{lstlisting}[language=JavaScript, caption=Emphasizing text using the HTML editing API, label=lst:execcommand-italics]
document.execCommand('italic', false, null);
\end{lstlisting}

\noindent will wrap the currently selected text inside an element in editing mode with \texttt{<i>} tags. The method accepts three parameters. The first paramter is the ''Command Identifier'', that determines which command to execute. For instance, this can be \texttt{italic} to italicize the current selection or \texttt{createLink} to create a link with the currently selected text as label.

\begin{lstlisting}[language=JavaScript, caption=Creating a link using the HTML editing API, label=lst:execcommand-link]
document.execCommand('createLink', false, 'http://google.de/');
\end{lstlisting}

The \textit{third} parameter will be passed on to the internal command given as first parameter. In the case of a \texttt{createLink} command, the third parameter is the URL to be used for the link to create. The \textit{second} paramter determines if executing a command should display a user interface specific to the command. For instance, using the \texttt{createLink} command with the second parameter set to \texttt{true} and not passing a third parameter, the user will be prompted with a system dialog to enter a URL. A full list of possible command identifiers can be found on MSDN\footnote{\url{https://msdn.microsoft.com/en-us/library/ms533049(v=vs.85).aspx}, last checked on 07/10/2015}.



\section{Usage of HTML Editing APIs for rich-text editors}
\label{sec:useage-of-html-editing-apis}

Most rich-text editors use HTLM editing APIs as their basis. CKEditor and TinyMCE dynamically create an \texttt{iframe} on instantiation and set its \texttt{body} to editing mode using \texttt{contenteditable}. Doing so, users can type inside the \texttt{iframe} so it effectively acts as text input field. Both libraries wrap the \texttt{iframe} in a user interface with buttons to format the \texttt{iframe}'s contents. When a user clicks on a button in the interface \texttt{document.execCommand} will be called on the \texttt{iframe}'s \texttt{document} and the selected text will be formatted. While using an \texttt{iframe} is still in practice, many newer editors use a  \texttt{div} instead. The advantages and disadvatanges of this technique will be discussed in Sections XY.