%!TEX root = ../../thesis.tex


\section{Browser support}
%% \section{Development of HTML Editing APIs}

With the release of Internet Explorer 5.5 and the introduction of editing capabilities, Microsoft released a short documentation\footnote{\url{https://msdn.microsoft.com/en-us/library/ms537837(VS.85).aspx}, last checked on 07/10/2015}, containing the attributes' possible values and element restrictions along with two code examples. Although a clear purpose has not been stated, the code examples demonstrated how to implement rich-text input fields with it. Mark Pilgrim, author of the ''Dive into'' book series and contributor to the the Web Hypertext Application Technology Working Group (WHATWG), also states that the API's first use case has been for rich-text editing\footnote{\url{https://blog.whatwg.org/the-road-to-html-5-contenteditable}, last checked on 07/10/2015}. 

%% With the release of Internet Explorer 5.5 and the introduction of editing capabilities, Microsoft released a sparse documentation\footnote{\url{https://msdn.microsoft.com/en-us/library/ms537837(VS.85).aspx}, last checked on 07/10/2015} describing only the availability and the before-mentioned element restrictions of these attributes. 

%% According to Mark Pilgrim, author of the ''Dive into'' book series and contributor to the the Web Hypertext Application Technology Working Group (WHATWG),, but its first use case has been for rich-text editing\footnote{\url{https://blog.whatwg.org/the-road-to-html-5-contenteditable}, last checked on 07/10/2015}. 

% It is notable, that the available command identifiers mostly include text-editing (or related) commands, but not exclusively. Other commands include navigating to other URLs or controlling the browser's cache.

In March 2003, the Mozilla Foundation introduced an implementation of Microsoft's designMode, named Midas, for their release of Mozilla 1.3. Mozilla already named this ''rich-text editing support'' on the Mozilla Developer Network (MDN)\footnote{\url{https://developer.mozilla.org/en/docs/Rich-Text\_Editing\_in\_Mozilla}, last checked on 07/10/2015}. In June 2008, Mozilla added support for contentEditable IDL and contenteditable content attributes with Firefox 3. 

Mozilla's editing API mostly resembles the API implemented for Internet Explorer, however, to this present day, there are still differences (compare\cite{ad}\cite{am}). This includes the available command identifiers\cite{am}\cite{ad} as well as the markup generated by invoking commands\cite{ai}. 

%\footnote{\url{https://msdn.microsoft.com/en-us/library/hh772123(v=vs.85).aspx}, last checked on 07/10/2015}
%\footnote{\url{https://developer.mozilla.org/en-US/docs/Midas}, last checked on 07/10/2015}
%\footnote{\url{https://developer.mozilla.org/en-US/docs/Midas}, last checked on 07/10/2015}
%\footnote{\url{https://msdn.microsoft.com/en-us/library/ms533049(v=vs.85).aspx}, last checked on 07/10/2015}
%\footnote{\url{https://developer.mozilla.org/en/docs/Rich-Text\_Editing\_in\_Mozilla#Internet\_Explorer\_Differences}, last checked on 07/10/2015}

%% Mozilla's command identifiers are restricted to text-editing command, showing the clear purpose of this API.

%% This may show, that even though rich-text editing was its first use case and Mozilla implemented it naming it that, this editing API was not originally intended to be used as such.

In June 2006, Opera Software releases Opera 9\cite{ap}, providing full support for contentEditable and designMode\cite{aq}, followed by Apple in March 2008\cite{ar} providing full support Safari 3.1\cite{caniuse_contenteditable}. MDN lists full support in Google Chrome since version 4\cite{as}, released in January 2010\cite{at}.

%\footnote{\url{http://www.opera.com/docs/changelogs/windows/}, last checked on 07/10/2015}
%\footnote{\url{http://www.opera.com/docs/changelogs/windows/900/}, last checked on 07/10/2015}
%\footnote{\url{https://www.apple.com/pr/library/2008/03/18Apple-Releases-Safari-3-1.html}, last checked on 07/10/2015}
%\footnote{\url{http://caniuse.com/#feat=contenteditable}, last checked on 07/10/2015}
% \footnote{\url{https://developer.mozilla.org/en-US/docs/Web/Guide/HTML/Content\_Editable}, last checked on 07/10/2015}
%\footnote{\url{http://googlechromereleases.blogspot.de/2010/01/stable-channel-update\_25.html}, last checked on 07/10/2015}

%In March 2008, Apple released Safari 3.1\footnote{\url{https://www.apple.com/pr/library/2008/03/18Apple-Releases-Safari-3-1.html}, last checked on 07/10/2015} including full support for contentEditable and designMode\footnote{\url{http://caniuse.com/#feat=contenteditable}, last checked on 07/10/2015}, followed by Opera Software in June 2006\footnote{\url{http://www.opera.com/docs/changelogs/windows/}, last checked on 07/10/2015} providing full support in Opera 9\footnote{\url{http://www.opera.com/docs/changelogs/windows/900/}, last checked on 07/10/2015}. MDN lists full support in Google Chrome since version 4\footnote{\url{https://developer.mozilla.org/en-US/docs/Web/Guide/HTML/Content\_Editable}, last checked on 07/10/2015}, released in January 2010\footnote{\url{http://googlechromereleases.blogspot.de/2010/01/stable-channel-update\_25.html}, last checked on 07/10/2015}.

Starting in November 2004, WHATWG members have started actively discussing to incorporate these editing APIs in the HTML5 standard. Through reverse engineering, the WHATWG developed a specification based on Microsoft's implementation\cite{ah} and finally decided to include it in HTML5. With W3C's coorporation and the split in 2011, similar editing APIs based on this work are now included in W3C's HTML5 Standard\cite{aw} and WHATWG's HTML Standard\cite{av}.

%\footnote{\url{https://blog.whatwg.org/the-road-to-html-5-contenteditable}, last checked on 07/15/2015}
%\footnote{\url{http://www.w3.org/TR/html5/editing.html}, last checked on 07/15/2015}
% \footnote{\url{https://html.spec.whatwg.org/multipage/interaction.html#editing-2}, last checked on 07/15/2015}