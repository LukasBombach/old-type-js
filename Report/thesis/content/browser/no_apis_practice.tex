%!TEX root = ../../thesis.tex
\section{Rich-text without HTML editing APIs in practice} 

Google completely rewrote their document editor in 2010 abandoning HTML editing APIs entirely. In a blog post\footnote{\url{http://googledrive.blogspot.fr/2010/05/whats-different-about-new-google-docs.html}, last checked on 07/18/2015}, they stated some of the reasons discussed in section \refsection{sec:disadvantages_of_html_editing_apis}. They state, using the editing mode, if a browser has a bug in a particular function, Google won't be able to fix it. In the end, they could only implement ''least common denominator of features''. Furthermore, abandoning HTML editing APIs enables features otherwise impossible, for example tab stops for layouting\cite{bw}. With the Google document editor, Google demonstrates it is possible to implement a fully featured rich-text editor using only JavaScript without HTML editing APIs.

% However, fetching input and modifying text will not suffice to implement a text editor or even a simple text field. There is many more things, that need to be considered which will be discussed in chapter \ref{ch:concept}. %\refchapter 


%''Ace'' and ''CodeMirror'' demonstrate it is possible to mimic text-inputs with JavaScript to implement code editors. Rich-text editing is usually being implemented using HTML editing APIs. There are a few exceptions. 

Google's document editor is proprietary software and its implementation has not been documented publicly. Most rich-text editors still rely on HTML editing APIs. The editor ''Firepad''\footnote{\url{http://www.firepad.io/}, last checked 07/23/2015} is another exception. It is based on ''CodeMirror'', a web-based source code editor, and extends it with rich-text formatting. The major disadvantage of Firepad is its origin as a source code editor. It generates ''messy'' (non-semantic) markup with lots of control tags. It has a sparse API that is not designed for rich-text editing and has no public methods to format the text. It is to be noted that Google's document editor generates lots of control tags as well, but it is only used within Google's portfolio of office apps where it may not be necessary to create \textit{well-formatted}, semantic markup. A full list of rich-text editors using and not using HTML editing APIs can be found in tables \ref{table:editors-editing-mode} and \ref{table:editors-non-editing-mode}. % https://github.com/plotnikoff/HTE


%% In October 1998 the World Wide Web Consortium (W3C) published the ''Document Object Model (DOM) Level 1 Specification''. This specification includes an API on how to alter DOM nodes and the document's tree\footnote{\url{http://www.w3.org/TR/REC-DOM-Level-1/level-one-core.html}, last checked on 07/10/2015}. It provided a standardized way for changing a website's contents. With the implementations of Netscape's JavaScript and Microsoft's JScript this API has been made accessible to web developers.

%\section{Rich-text libraries implemented without editing APIs}


