%!TEX root = ../../thesis.tex

\section{Disadvantages of rich-text editing without editing APIs}

\subsection{Formatting}
\label{subsec:noapi_dis_formatting}
The HTML editing APIs' formatting methods take away a crucial part of rich-text editing. Especially on the web, where a text may come from various sources, formatting must account for many edge cases. Nick Santos, author of Medium's rich-text editor states:

\begin{quotation}
\textit{''Our editor should be a good citizen in [the ecosystem of rich-text editors]. That means we ought to produce HTML that's easy to read and understand. And on the flip side, we need to be aware that our editor has to deal with pasted content that can't possibly be created in our editor. \cite{medium_ce_terrible}''}
\end{quotation}

%\footnote{\url{http://stackoverflow.com/questions/11240602/paste-as-plain-text-contenteditable-div-textarea-word-excel/11290082#11290082}, last checked on 07/13/2015}

An editor implemented \textit{without} HTML editing APIs does not only need to account for content (HTML) that will be pasted into the editor\footnote{Medium uses HTML editing APIs} (in fact, content can be sanitized before it gets inserted in the editor, see \refsection{sec:impl_pasting}), but also for content that will be loaded on instantiation. It cannot be assumed that the content that the editor will be loaded with (for example integrated in a CMS), is \textit{well-formatted} markup or even valid markup. ''Well-formatted'' means, the markup of a text is \textit{simple} in the sense that it expresses semantics with as few tags as possible (and it conforms the standards  of the W3C). In HTML, the same visual representation of a text, can have many different---and valid---underlying DOM representations. Nick Santos gives the example of the following text\cite{so_paste_plain}:

%\footnote{\url{http://stackoverflow.com/questions/11240602/paste-as-plain-text-contenteditable-div-textarea-word-excel/11290082#11290082}, last checked on 07/13/2015}

\begin{quotation}
The \underline{hobbit} was a very well-to-do hobbit, and his name was \textbf{\textit{Baggins}}.
\end{quotation}

\noindent The word ''Baggins'' can be written in any of the following forms:

\begin{lstlisting}[language=html, caption=Different DOM representations of an equally formatted text, label=lst:different-dom-representations]
<strong><em>Baggins</em></strong>
<em><strong>Baggins</strong></em>
<em><strong>Bagg</strong><strong>ins</strong></em>
<em><strong>Bagg</strong></em><strong><em>ins</em></strong>
\end{lstlisting}

A rich-text editor must be able to edit any of these representations (and more). Furthermore, the same edit operation, performed on any of these representations must provide the same \textit{expected} behavior, i.e. generate the same visual representation and produce predictive markup. Above that, being a ''good citizen'' it should produce simple and semantically appropriate HTML even in cases when the given markup does not conform this rule.% It may even improve the markup it affects.


% Apparently implementing editing is prone to a lot of bugs issues **and edge-casese**. It can be assumed that it is difficult to do so. / Discuss the issues discussed by WHATWG here. It's difficult to do this.

% Also this can be seen as ''yet another'' implementation of contenteditable, equal to browser implementations. Just one more editor for developers to take care of. However, this isn't quite correct, cos developers usually can choose a single editor that is used for the entire project, so they only have to take care of a single editor.

\subsection{Native components}
\label{subsec:disadv_mimic_native}

As discussed in section \refsection{sec:html-editing-apis}, when an element is switched to editing mode using the HTML editing APIs, users can click inside the text and will be presented with a caret. They can move the caret with the arrow keys and enter text that will be inserted at the appropriate offset. They can use keyboard shortcuts and use the mouse's context menu to paste text. Behavior that is common for rich-text input, for instance that a new list item will be created when users press ''enter'' inside a list, is implemented by the browser. None of this is available when not using HTML editing APIs. All of this must be accounted for and implemented in JavaScript. Elements like the caret must be mimicked with DOM elements like the \code{div} element. The users' input must be read with JavaScript and either move the caret or modify the text.

%As described in section XY, rich-text editing consists of many components like the caret or the ability to paste text via a context menu. These basic components require complex implementations. HTML editing APIs offer these components natively without further scripting.

\subsection{Possible performance disadvantages}

Modifying the text on a website means manipulating the DOM. DOM operations can be costly in terms of performance as they can trigger a browser reflow\footnote{\url{https://developers.google.com/speed/articles/reflow}, last checked on 07/19/2015}. While it should be a goal to keep browser interactions to a minimum, there is no way to avoid DOM interaction with any visual text change.

%The same goes for mimicking some elements like the caret. To display a caret a DOM element like a \texttt{div} is needed and it needs to be moved by changing its style attribute. 

%\footnote{\url{https://developers.google.com/speed/articles/reflow}, last checked on 07/19/2015}

\subsection{File size} While bandwidth capacities have vastly improved, there may still be situations where a JavaScript libraries' file sizes matter. This may be for mobile applications or for parts of the world with less developed connections. When not using HTML editing APIs, a lot of code must be written and transmitted just to enable basic text editing, which would not be needed otherwise.