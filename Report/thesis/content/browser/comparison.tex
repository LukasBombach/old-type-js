%!TEX root = ../../thesis.tex

\section{Conclusion}

HTML editing APIs will generate different markup on most browsers, their functionality is limited and restricts web developers from extending it. As for the current state, their implementations contain plenty bugs on almost every system, which cannot be fixed by web developers. The ''DOM Level 1'' APIs required to perform the same tasks as HTML editing APIs have been developed and tested for more than 15 years and tend to be stable. Google's document editor demonstrates it is possible to implement a fully featured editor without using editing APIs. Doing so will avoid any restriction and limitation of these APIs and give web developers full control of all components, the generated markup and possible bugs.

% Chapter \refchapter{} and \refchapter{} demonstrate ways to implement 

%HTML editing APIs vs other method, pros and cons, conclusion. Also HTML editing APIs will get better, still the API sucks, fewer features, also still dependent on browser development in terms of possible bugs in the future and further features. Edge casese can be tackled and solutions rolled out immediately. Cross browser inconsistencies are less of a problem and can also be tackled as required.


%
%
%
%
%
%    NEHME ICH RAUS UND MACHE ICH IN DIE EVALUATION
%
%