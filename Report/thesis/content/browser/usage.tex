%!TEX root = ../../thesis.tex

\section{Discussion}
%\section{Discussion of HTML Editing APIs}

Understanding the history of the HTML editing APIs, the reasons for their wide browser support and their final standardization are questionable. It can be doubted if they fit their purpose secifically well. In fact, all major browsers mimicked the API as implemented in Internet Explorer 5.5, even though there was no specification for it. The reasons for this have not been publically discussed. A reason may have been to be able to compete with the other browsers. Both, Microsoft's original implementation as well as Mozilla's adoption have been released in the main years of the so-called ''browser wars''. Mozilla adopted Microsoft's API applying practically no change to it. It can be argued that this has been part of the struggle for market shares with Microsoft's Internet Explorer. At this time, it was essential for any browser to be able to be compatible with as many websites as possible. Many websites were only optimized for a specific browser only. To gain market share, it was essential to support methods that other browsers already offered and that have been used by the web developers. Being able to display websites just as good as their competitor may have been a key factor for Mozilla's decision to implement Microsoft's HTML editing APIs and not alter them in any way. Creating another standard would have been a disadvantage over the then stronger Internet Explorer in getting users to choose Mozilla. 

As discussed in section \refsection{sec:edit_apis_browser_support}, other now popular browsers, i.e. Chrome, Safari and Opera, implemented these APIs only years later, when JavaScript libraries based on them have already been popular and widely used, which can be seen as a reason for this decision. As described in section \refsection{sec:standardization-of-html-editing-apis}, it has clearly been stated, that the reason for standardizing these APIs for rich-text editing has been to ensure browser support.

The API itself stems from the time when the usage of the web was different from today, its future was still unknown and web applications like Google Docs have not even been thought of. It should be discussed if this API really is the answer to all problem and if it still fits (or ever fit) modern requirements for content management systems or web application. The advantantages, disadvantages and practical issues will be discussed in sections x y z.


%It had to be reverse-engineered\footnote{\url{https://blog.whatwg.org/the-road-to-html-5-contenteditable}}. 

% but because they were \textit{already there} and could used on most PCs.% by the majority of users.

%% From its history it's clear to see that the HTML editing APIs just came to be by adoption of browser manufacturer's and ultimately have been standardized for the main reason because they have already been there.


% Using an iFrame as input field has been popular for years and many other rich-text editors have worked this way. The reasons for this have been described by Tim Down, author of the popular Rangy library and Piotrek Koszuliński, core developer of CKEditor\footnote{\url{http://stackoverflow.com/a/4430583/1183252}, last checked on 07/16/2015}\footnote{\url{http://stackoverflow.com/a/11222149/1183252}, last checked on 07/16/2015}:

% \begin{enumerate} \item Styling is encapsulated in the \texttt{iframe} and does not affect the document outside the editor. \item Historically, browsers had buggy implementations of the \texttt{contenteditable}, so it was not possible to only make specific elements on a page editable and act as input field. \end{enumerate}

% Even today many editors still use this technique, but it has severe downsides and the pro arguments are obsolete. This will be discussed in Sections XY.

%% How Js Libraries work. Maybe how only a few work. StackOverlfow quote, buglists, Medium post, other stuff to find.

%% Seeing how editing APIs have come to existance, as an undocumented API by microsoft, taken over by mozilla, bossted by JS libraries and then adopted by other browsers, it can be understood how editing APIs came to be. However, by its introduction by microsoft, it has not been stated that this has been its original purpose. And even if Mozilla picked up on it, it is not clear that this API is in fact the best way to implement rich-text editing. WHATWG has discussed this API, arguing if it is the best way to provide rich-text functionality.

%% ** WICHTIG DEN BOGEN ZU SPANNEN WARUM ICH ÜBER DIE GESCHICHTE SCHREIBE** ERKLÄREN, DASS ALLE DIE API VON MICROSOFT ÜBERNOMMEN HABEN; ANSTATT RICH-TEXT ELEMENTE EINZUFÜHREN. HTML5 HAT KOMPONENTEN FÜR TIME UND DATE; WARUM NICHT RICH TEXT- WHATWG HAT DAS DISCUSSED. WAS WAREN DIE GRÜNDE?. BEI DER BESCHREIBUNG DARAUF ACHTEN DASS ICH DEN GRUND GEBE; WARUM ICH DIE ENTWICKLUNG SO LANGE ERKLÄRE. VIELLEICHT AUCH SAGEN, DASS TROTZ DASS HTML5 ERST SEIT X RELEASED WURDE, ANHAND DER GESCHICHTE GESEHEN WERDEN KANN; DASS ES SCHON LANGE BROWSER SUPPORT GIBT.

%% Seeing the history of editing APIs, it is understandable how this has become the standard for rich-text editing. However, with its introduction in Internet Explorer 5.5, it has not been stated that the \texttt{designMode} end \texttt{contentEditable} attributes have been intended to enable rich-text editing. Sections X and Y will discuss the advantages and disadvantages of these APIs.

% Rich-text editing in browsers is only possible though JavaScript. Essentially, libraries enabling rich-text editing display a nested webpage through an iFrame and let the user modify its contents to emulate a rich-text input. Commonly, modification is realized though the browser's so-called ''HTML Editing APIs'', which will be discussed in Section XXX (HTML Editing APIs).