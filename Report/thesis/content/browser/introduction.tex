%!TEX root = ../../thesis.tex

\section{Overview}

Developing a text editor in a browser environment differs from implementing a text editor in a desktop environment. Any implementation is based on the axioms of the browsers. Development is generally restricted to the components and APIs offered by the HTML5 standard and experimental features that are usually implemented in a subset of browsers\footnote{If a specific component or API can be used is determined by the intended audience and browser market share.}. However, the boundaries of these restrictions can be pushed. It is common practice to combine native elements and APIs in ways they have not been designed for to enable features not natively offered. These techniques are often referred to as ''hacks'' and, despite their terminology, are generally not regarded as a bad practice.

This chapter will discuss the basics of text and rich-text editing in browsers as well as the APIs and techniques that are natively offered.%as well as the specifics of implementing a rich-text editor in the browser without the use of hacks.

% by modern web browsers

%\footnote{Native to the browser, not the operating system.}

%This chapter will describe the basics of text and rich-text editing in browsers as well as the APIs and techniques used. The origins and the history of rich-text editing pose the question if the paradigms rich-text editing is based on have been thoroughly reviewed and if alternative ways for an implementation, possibly using hacks, could be considered.

% When not using third-party plugins like Adobe Flash or Microsoft Silverlight, development is restricted to the components and APIs
