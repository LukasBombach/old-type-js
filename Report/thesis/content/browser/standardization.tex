%!TEX root = ../../thesis.tex

\section{Standardization of HTML Editing APIs}
\label{sec:standardization-of-html-editing-apis}

% It makes sense to use HTML editing APIs for rich-text editing. 

% Microsoft's demos, that have been published with the release of these APIs, suggested this application. Mozilla mimicked this API and called their implementation ''rich-text editing API''. Other browsers followed with APIs based on Microsoft's idea of editable elements. 

%HTML editing APIs, although not standardized, have been the \textit{de facto} standard for implementing rich-text editors on the web. It can be imagined, that browsers could have implemented a different API or even dedicated rich-text input elements. Such an element could also have been introduced with new standards.

%It would have been possible to offer a native user interface component, a dedicated rich-text input field, instead of this API.

HTML editing APIs have been the \textit{de facto} standard for implementing rich-text editors on the web, but have only been standardized in October 2014 with HTML5. 

HTML5 introduces 13 new types of input fields \cite{HTML5}. It can be imagined that along with these elements, the standard could have introduced a native rich-text input element as well, but none of the elements comprises such capabilities. The WHATWG, the working group that mainly developed the HTML5 standard, discussed this issue publicly. The problems that have been faced with that idea are as follows:

%The WHATWG discussed various ways to specify rich-text editing for the upcoming HTML5 standard, including dedicated input fields. The issues that have been faced with that idea are as follows:

\begin{enumerate} 
\item Finding a way to tell the browser which language the rich-text input should generate. E.g. should it output BBCode\footnote{A then popular markup language for bulletin boards}, (X)HTML, Textile or something else?
\item How can browser support for a rich-text input be achieved?
\end{enumerate}

%% Den browser support teil kann ich später gut aufgreifen, damit dass meine library immer 100% browser support hat

Ian Hickson, editor of WHATWG and main author of the HTML5 specification, addresses these main issues in a message from November 2004\footnote{\url{https://lists.w3.org/Archives/Public/public-whatwg-archive/2004Nov/0014.html}, last checked on 07/16/2015}. He states

\begin{quotation}
\textit{''Realistically, I just can't see something of this scoped[sic] [the ability to specify a language for a rich-text input and possibly to specify a subset of language elements allowed] getting implemented and shipped in the default install of browsers.''}
\end{quotation}

and agrees with Ryan Johnson, a contributor to the standard, who states

\begin{quotation}
\textit{''Anyway, I think that it might be quite a jump for manufacturers. I also see that a standard language would need to be decided upon just to describe the structure of the programming languages. Is it worth the time to come up with suggestions and examples of a programming language definition markup, or is my head in the clouds?''}
\end{quotation}

Ian Hickson finally concludes

\begin{quotation}
\textit{''Having considered all the suggestions, the only thing I could really see 
as being realistic would be to do something similar to (and ideally 
compatible with) IE's "contentEditable" and "designMode" attributes.''}
\end{quotation}

Mark Pilgrim lists this as a milestone of the decision to integrate Microsoft's HTML editing APIs in the standard of the WHATWG.\footnote{\url{https://blog.whatwg.org/the-road-to-html-5-contenteditable}, last checked on 07/16/2015}. In cooperation with the W3C, the work by the WHATWG, including the standardization of the editing APIs, have been incorporated in the HTML5 standard. The cooperation between the WHATWG and the W3C ended in Juli 2012\footnote{\url{http://lists.w3.org/Archives/Public/public-whatwg-archive/2012Jul/0119.html}, last checked on 07/16/2015}, which led the WHATWG to publish and maintain an own standard, the ''HTML Living Standard'' \cite{HTMLWHATWG} that includes the same specifications on HTML editing APIs as HTML5.

% The WHATWG incorporated these APIs. 

%HTML editing APIs have been standardized in the HTML5

% By understanding the origins, the development and the process of standardization, it can be seen that the incorporation of the HTML editing APIs as designed by Microsoft has not been decided because they are a \textit{good idea}, but to be compatible with other systems and ultimately Internet Explorer 5.5.

