\section{Introduction}

During the letterpress era, ''marking up'' text has been the profession of adding formalized annotations to a text, that described the structure and formattings of the document. The annotated document was given to a typesetter to follow the instructions and use a movable type system to print the document accordingly.

\section{History of markup languages}

With the computerization of typesetting, so-called ''markup languages'' have been invented that embedded the annotations in the text that was given to the typesetters. Coombs, Renear, and DeRose describe six types of markup languages: Punctuational, presentational, procedural, descriptive, referential and metamarkup \cite{Coombs:1987:MSF:32206.32209}. Punctuational markup solely refers to the use of punctuation to structure text, referential markup describes the ability of a markup language to refer to other documents and a metamarkup language can be used to describe other markup languages. Procedural markup includes commands for a computer program on how to render the text step by step. Presentational markup contains specific descriptions on the formatting of a text, describing particular parts as italicized, bold, indented etc. Descriptive markup describes the elements of a text as types. For instance, a part of a text can be marked to be a quote or a headline, but there would be no definition on how these elements should be displayed. A renderer can parse the document and present it with specific styles. Watson describes the difference between descriptive and presentational markup as generic and specific markup. Generic markup only describes the structure of a document while specific markup explicitly describes its styling \cite{watsonhistory}. 

The invention of generic markup is credited to William Tunnicliffe who proposed his ideas at a meeting at the Canadian
Government Printing Office in 1967 \cite{Goldfarb:1991:SH:102902}. Markup that was given to typesetters still needed to be translated for the particular typesetting system that was used. This led to to higher costs and the need for a standard emerged \cite{watsonhistory}. In the late 60s, Stanley Rice, a book designer, proposed this idea of generic markup to the Graphic Communications Association (GCA), which formed the GCA GenCode committee to work on a standard for generic markup \cite{Goldfarb:1991:SH:102902}. This work has been authorized by the Organization for Szandardization (ISO) and the American National Standards Institute (ANSI). In 1969 Charles Goldfarb, Edward Mosher, and Raymond Lorie invented the Generalized Markup Language (GML) \cite{watsonhistory}, a generic markup language for IBM. Goldfarb later maintained the cooperation of ISO, ANSI and the GCA GenCode committee and in 1985 drafted the proposal for the ''Standard Generalized Markup Language'' (SGML), the first international standard for a generic markup language. SGML was based on the work of the GCA GenCode committee as well as the GML \cite{Goldfarb:1991:SH:102902} and published in 1986 as ISO 8879:1986 \cite{ISO8879}.

%Markup that was given to typesetters still needed to be translated for the particular typesetting system that was used. This led to to higher costs and the need for a standard emerged. In 1969 Charles Goldfarb, Edward Mosher, and Raymond Lorie invented the Generalized Markup Language (GML) \cite{watsonhistory}, a generic markup language for IBM. During the late 60s, Stanley Rice a prominent book designer

% Markup that was given to typesetters still needed to be translated for the particular typesetting system that was used. This led to to higher costs and the need for a standard emerged. In the late 60s and early 70s William W. Tunnicliffe and Stanley Rice independently

% In the late 60s, markup that was given to typesetters still needed to be translated for the particular typesetting system that was used. This led to to higher costs and the need for a standard emerged.

%In contrast to the presentational markup, a part of a text can be marked as a quote or as emphasized, without styling it, while a presentational markup would give that same text a specific style. A renderer of the descriptive document can decide on how to format the emphasized text.
%Generic markup only describes the structure of the document 

%These types are not necessarily exclusive to one another. Most languages use punctuation and a markup language can be descriptive but also include presentational descriptions.



% Since there was no standardized format yet, these languages still needed to be translated for the particular typesetting system.

%markup languages---languages describing what has previously been expressed with annotations---have been invented, that were part of the text that was given to the typesetters.



\section{HyperText Markup Language}

\label{sec:html_sgml_def}

HyperText Markup Language (HTML) is an implementation of SGML \cite[SGML and HTML]{HTML401}. One of the primary functions of web browsers is to display HTML formatted sources as visually formatted text. HTML uses tags to specify the contents of a document. Being an instance of SGML it mostly uses generic tags to define headlines, paragraphs or quotations inside a document, but also allows for specific tags, defining parts of the contents as italicized or bold.

Tags are strings inside the document's text that itself are delimited by the ''<'' and ''>'' characters. \reflisting{lst:html_markup_bold_example} demonstrates the HTML required to render the following text:

\begin{quotation}
In a hole in the ground there lived a \textbf{hobbit}
\end{quotation}

\begin{lstlisting}[language=html, caption=Text formatted as bold with the ''strong'' tag, label=lst:html_markup_bold_example]
In a hole in the ground there lived a <strong>hobbit</strong>
\end{lstlisting}

The word ''hobbit'' must be enclosed with a ''strong'' start and end tag. Start and end tags are distinguished by adding a solidus to the end tag. HTML defines 127 tags to format document contents as well as to add metadata about the document itself \cite{mozel}.

By the recommendation of the World Wide Web Consortium (W3C), browsers must represent a document marked up with HTML with the Document Object Model (DOM) \cite{DOM1}, a tree structure containing every tagged element and its texts as nodes. The specification of the DOM defines an API to manipulate it and change the contents of a website dynamically.

%\section{Rich text in browser environments}

%One of the primary functions of web browsers is to display rich text. Browsers load documents formatted with the HyperText Markup Language (HTML) from remote sources and display them as visually formatted text to the user.
%Therefore the HTML source code, provided as a string of characters, must be parsed and rendered accordingly. HTML is a markup language, that means it uses tags to specify which parts of the document should be rendered in a particular formatting. Tags can either encapsulate parts of the text with a start and an end tag or, in some cases, be used as standalone entities. 
%The latter is used to insert ''void elements'', which are either visible elements that cannot have nested contents, like line breaks, images and horizontal rules or elements that act as attributes of other elements. Furthermore, standalone tags can be ''foreign elements'' for which nested contents are omitted, like a Scalable Vector Graphics (SVG) that has no contents \cite[Elements]{HTML5}. 
%To format a string of characters, it must be encapsulated in a start and end tag that describe which formatting to apply. Tags are strings inside the document's text that itself are delimited by the ''<'' and ''>'' characters. \reflisting{lst:html_markup_bold_example} demonstrates the HTML required to render the following text:

%\begin{quotation}
%In a hole in the ground there lived a \textbf{hobbit}
%\end{quotation}

%\begin{lstlisting}[language=html, caption=Text formatted as bold with the ''strong'' tag, label=lst:html_markup_bold_example]
%In a hole in the ground there lived a <strong>hobbit</strong>
%\end{lstlisting}

%The word ''hobbit'' must be enclosed with a ''strong'' start and end tag. Start and end tags are distinguished by adding a solidus to the end tag. HTML defines 127 tags to format document contents as well as to add metadata about the document itself \cite{mozel}.% and consists entirely of ''generic markup''.

%\section{Types of markup}

%3 typen von markup

%html = generisches deklaratives markup

%jetzt kommt die geschrichte


%To format a word in a strong font, it must be encapsulated in a start tag and an end tag
%parse and view 
%Web browsers are hypertext markup language viewers


