\section{Introduction}

During the letterpress era, ''marking up'' text has been the profession of adding formalized annotations to a text, that described the structure and formattings of the document. The annotated document was given to a typesetter to follow the instructions and use a movable type system to print the document accordingly.

\section{History of markup languages}

With the computerization of typesetting, languages have been invented that included the annotations in the text that was given to the typesetters. These languages still needed to be translated for the particular typesetting system that was used.  \cite{watsonhistory}.

% Since there was no standardized format yet, these languages still needed to be translated for the particular typesetting system.

%markup languages---languages describing what has previously been expressed with annotations---have been invented, that were part of the text that was given to the typesetters.


HTML is an implementation of the Standard Generalized Markup Language (SGML) \cite{ISO8879}.

\section{Rich text in browser environments}

One of the primary functions of web browsers is to display rich text. Browsers load documents formatted with the HyperText Markup Language (HTML) from remote sources and display them as visually formatted text to the user.
Therefore the HTML source code, provided as a string of characters, must be parsed and rendered accordingly. HTML is a markup language, that means it uses tags to specify which parts of the document should be rendered in a particular formatting. Tags can either encapsulate parts of the text with a start and an end tag or, in some cases, be used as standalone entities. 
The latter is used to insert ''void elements'', which are either visible elements that cannot have nested contents, like line breaks, images and horizontal rules or elements that act as attributes of other elements. Furthermore, standalone tags can be ''foreign elements'' for which nested contents are omitted, like a Scalable Vector Graphics (SVG) that has no contents \cite[Elements]{HTML5}. 
To format a string of characters, it must be encapsulated in a start and end tag that describe which formatting to apply. Tags are strings inside the document's text that itself are delimited by the ''<'' and ''>'' characters. \reflisting{lst:html_markup_bold_example} demonstrates the HTML required to render the following text:

\begin{quotation}
In a hole in the ground there lived a \textbf{hobbit}
\end{quotation}

\begin{lstlisting}[language=html, caption=Text formatted as bold with the ''strong'' tag, label=lst:html_markup_bold_example]
In a hole in the ground there lived a <strong>hobbit</strong>
\end{lstlisting}

The word ''hobbit'' must be enclosed with a ''strong'' start and end tag. Start and end tags are distinguished by adding a solidus to the end tag. HTML defines 127 tags to format document contents as well as to add metadata about the document itself \cite{mozel}.% and consists entirely of ''generic markup''.

\section{Types of markup}

3 typen von markup

html = generisches deklaratives markup

jetzt kommt die geschrichte


%To format a word in a strong font, it must be encapsulated in a start tag and an end tag
%parse and view 
%Web browsers are hypertext markup language viewers


