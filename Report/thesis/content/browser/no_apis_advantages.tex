%!TEX root = ../../thesis.tex

\section{Advantages of rich-text editing without editing APIs}

With a pure JavaScript implementation, many of the problems that HTML editing APIs have, can be solved. The issues discussed in \refsection{sec:disadvantages_of_html_editing_apis} will be addressed hereinafter.

\subsection{Generated output and flawed API} 
\label{subsec:adv_flawed_api}

 % an editor can be implemented to allow developers using the editor to \textit{choose} the output

The generated markup, if implemented through JavaScript and DOM Level 1 methods, can be chosen with the implementation of the editor. Furthermore, the decision of the generated output can be given to the developers working with the editor. Section \refsection{sec:disadvantages_of_html_editing_apis} describes the inconsistent output across various browsers as well as the restrictions of the API design of \texttt{execCommand}. Both issues can be adressed by offering a method to wrap the current selection in arbitrary markup. jQuery's \texttt{htmlString} implementation\footnote{\url{http://api.jquery.com/Types/\#htmlString}, last checked on 07/19/2015} demonstrates a simple and stable way to define markup as a string and pass it as an argument to JavaScript methods. A sample call could read as follows.

\begin{lstlisting}[language=JavaScript, caption=Example calls to format text, label=lst:format-examples-api-alternative]
// Mimicking document.execCommand('italic', false, null);
editor.format('<em />');

// Added functionality
editor.format('<span class="highlight" />');
\end{lstlisting}

This will allow developers to choose which markup should be generated for italicising text. The markup will be consistent in the scope of their project. Since the DOM manipulation is implemented in JavaScript and not by high-level browser methods, this will also ensure the same output across all systems and solve cross-browser issues. The second example function call in listing \ref{lst:format-examples-api-alternative} demonstrates that custom formatting, fitting the needs of a specific project, can be achieved with the same API, giving developers a wider functionality.

\refsubsec{subsec:disadv_mimic_native} discusses the disadvantage, that when not using HTML editing APIs, native components like the caret or the text input must be implemented with JavaScript as they are not provided without using HTML editing APIs. On the flip side, this allows full control over these components that can be exposed via an API to other developers.

%As discussed in section AB, many components native to text editing have to be implemented in JavaScript. This requires some effort but also enables full control and direct over it. Ultimately, these components can be exposed in an API to other developers, enabling options for developing editors, not offered by HTML editing APIs. An example API will be discussed in sectionXImplementationn.

\subsection{Restrictions} When implementing an editor in pure JavaScript, the limitations imposed by the HTML editing APIs, do not apply. Anything that can be implemented in a browser environment can also be implemented as part of a rich-text editor. The Google document editor demonstrates rich functionality that would not be possible with an implementation based on HTML editing APIs.%including layouting tools or floating images. Both are features are hardly possible in an editing mode enabled environment\footnote{\url{http://googledrive.blogspot.fr/2010/05/whats-different-about-new-google-docs.html}, last checked on 07/19/2015}. %SectionABC discusses some use cases exploring the possibilites of rich-text editing implemented this way. % etherpad, markdown etc

\subsection{Clipboard}

Without a native text input or an element switched to editing mode with HTML editing APIs, clipboard functionality is not available. Users cannot paste contents from the clipboard unless one of these elements is focused. However chapter \refchapter{ch:impl} demonstrates a way that not only allows clipboard support, but also grants full control over the pasted contents.

% In a pure JavaScript environment, clipboard functionality seems to be harder to implement than with the use of editing mode. Apart from filtering the input, pasting is natively available---via keyboard shortcuts as well as the context menu. However, as demonstrated in section IMPLEMENTATION, it is possible to enable native pasting---via keyboard and context menu---even without editing mode. Furthermore, it is possible to filter the pasted contents before inserting them in the editor.

\subsection{Bugs} %No software can be guaranteed to be bug free. However, 

By refraining from using HTML editing APIs, all of its numerous bugs will be avoided. An the implementation can be aimed to minimize interaction with browser APIs, especially unstable or experimental interfaces. DOM manipulation APIs have been standardized for more than 15 years and tend to be well-proven and stable. Bugs that occur will mostly be part of the libary and can be fixed and not only worked around. Bug fixes can be rolled out to users when they are fixed. This will free development from being dependent on browser development, update cycles and user adoption. 

%  minimize the number of ''unfixable'' bugs and ultimately

% This means that, other than with HTML editing APIs, bugs that occur are part of the library can be fixed and not only worked around. Furthermore, with minimizing browser interaction, bugs probably occur indeopendently of the browser used, which makes finding and fixing bugs easier.

%By refraining from using HTML editing APIs developing an editor will be independent from all of the APIs' bugs. Going a step further, 


% It puts the contenteditable implementation in the hand of JavaScript developers. We no longer have to wait for browsers to fix issues *and conform each other* and thus can be faster, at least possibly, than browsers are.

% An implementation without editing APIs cannot guarantee to be bug free, but not using these APIS, using only well-proven APIs and minimizing interaction with browser APIs will put the development and the fixing of bugs into the hands of JavaScript delopers. In other words, we \textit{can} actually \textit{fix} bugs and do not have to work around them and wait for browsers and browser usage to change (both takes long time).