%!TEX root = ../../thesis.tex

\section{Treating HTML editing API related issues}
\label{sec:ed_api_treating}

Since the issues arising with HTML editing APIs are part of the browser's implementation, they cannot be fixed by JavaScript developers. The common approach for most rich-text editors is to use HTML editing APIs and wrap it in a library while using workarounds for its issues and bugs internally. It is to be noted, as Piotrek Koszuli\'{n}ski points out, that the majority of rich-text editors ''really do not work'' \cite{bj}. This is usually the case when the problems discussed in \refsection{sec:disadvantages_of_html_editing_apis} have not been addressed and the library solely consists of a user interface wrapping an element in editing mode.

% \footnote{\url{http://stackoverflow.com/questions/10162540/contenteditable-div-vs-iframe-in-making-a-rich-text-wysiwyg-editor/11479435\#11479435}}

Having to account for multiple browser implementations, working around bugs can result in a big file size and a complex architecture. Most edge cases can only be learned from experience, not be foreseen or analyzed by debugging source code. %Piotrek Koszuli\'{n}ski writes ''We [...] are working on this for years and we still have full bugs lists'' \cite{sopp}.

In practice, there are a few attempts to implement pure wrappers that will take care of the beforementioned issues, to support other developers with a working api. This approach is generally not well-adopted though. In general, most libraries are distributed as independent editors implementing their own solutions for addressing these issues---or are forks of other editors implemented with a different user interface.

Subsections \refsubsec{subsec:treating_issues_first} through \refsubsec{subsec:treating_issues_last} will discuss some approaches to treat the beforementioned issues.


%There are various approaches to implement workarounds. Some libaries attempt to wrap HTML editing APIs and treat bugs and inconsitent behaviour internally. This apporach is generally not well-adopted. The most popular libraries related to web-based rich-text editing, rated by the numbers of ''stars'' given, are distributed as rich-text-editing user interface components (i.e. rich-text editors).



\subsection{HTML output} 
\label{subsec:treating_issues_first}

Editors like CKEditor offer some configuration on the generated HTML output\footnote{\begin{sloppypar}\burl{http://docs.ckeditor.com/\#!/guide/dev\_output\_format-section-adjusting-output-formatting-through-configuration}, last checked on 08/19/2015\end{sloppypar}}, but in the case of CKEditor this is very limited. The underlying issue is that HTML editing APIs cannot be configured. The only way to work around this issue is to implement custom methods to apply formatting in JavaScript and not using the \code{execCommand} interface\footnote{HTML editing APIs can still be used for text input and other functionality}. The proprietary ''Redactor Text Editor'' demonstrates such an implementation.

Medium.com takes a different approach and implements an extensive framework that will compare the markup of the editor with a model of the visual representation that the markup generates and corrects the DOM on each change\footnote{\url{https://medium.com/medium-eng/why-contenteditable-is-terrible-122d8a40e480}, last checked 08/19/2015} to conform a defined norm.

\subsection{Flawed API}

HTML editing APIs are usually wrapped in the API of an editor, that offers more functionality than the original API. \code{execCommand} offers the \code{insertHTML} command that allows inserting custom elements. As discussed in the previous paragraph, extending the formatting capabilities requires a JavaScript implementation.

\subsection{Clipboard}

For browsers, that do not offer native support to control and process the contents pasted from the clipboard, workarounds must be used. There are two approaches to this

\begin{enumerate}
\item Sanitize the editor's contents after a paste event.
\item Proxy a paste event to insert its contents into another element and read the contents from it.
\end{enumerate}

The ''Redactor Text Editor'' uses the first approach. While reading contents from the a paste event is not fully supported, the event itself will be triggered by all major browsers, even most older versions\footnote{That have a market share BETTER CHECK THIS AGAIN}. Once the event has finished and the contents have been inserted to the editor, a ''cleaned up'' procedure can remove unwanted contents.

CKEditor and TinyMCE have been developed before most major browsers supported clipboard events. Both editors implement a technique to permit pasting formatted text, that has been the standard for many years. CKEditor and TinyMCE create a hidden \code{textarea} element and listen for common ''paste'' keyboard shortcuts (\keystroke{ctrl}\keystroke{v} and \keystroke{shift}\keystroke{ins}). When a user presses these keys, the hidden \code{textarea} will be focused and thereby be the target in which the browser will paste the clipboard's contents. After a short delay, the editors can read the \code{textarea}'s contents. Since \code{textarea} elements allow plain text only, the contents will be removed of any formatting and can then be inserted to the editor. However, this does not account for pasting from the context menu. For this CKEditor overrides the native context menu with a custom menu containing a custom ''paste'' menu item, that will open a modal instructing the user to paste his or her contents using the keyboard shortcuts. TinyMCE overrides the native context menu too, but does not display a paste option. Up to the current versions CKEditor\footnote{CKEditor 4.5.1 } and TinyMCE \footnote{TinyMCE 4.2.3}, this is still the case.

CodeMirror\footnote{\url{https://codemirror.net/}, last checked on 08/22/2015}, a web-based source code editor enhances this approach by moving the textarea to the cursor's position when the user presses his or her right mouse button. This way a native context menu can be displayed while the paste option would insert the clipboard's contents to a designated \code{textarea}, that can be read from.

On the downside, the paste event cannot be proxied to the \code{textarea} if the user uses the browser's menu bar to paste contents.

%To control and sanitize the contents pasted by from the clipboard, editors take various approaches. Since not all browsers support listening to paste events and reading its contents, some editors and focus a hidden textarea, that the contents will then be pasted into by the browser. After a delay, the editor reads the contents from the textarea, which can then be processed and inserted to the editor's contents. CKEditor and TinyMCE enable the option to restrict pasting to unformatted text this way. 

\subsection{Bugs}

Generally, bugs cannot be fixed. The only way to treat bugs in browsers is by avoiding them, shimming them with JavaScript methods or ''cleaning up'' after they have occurred.

\subsection{Restrictions}
\label{subsec:treating_issues_last}

The restrictions the HTML editing API imposed on the contents of the editor is an even bigger problem. Taking the example of layouting with tab stops\footnote{\url{http://googledrive.blogspot.fr/2010/05/whats-different-about-new-google-docs.html}, last checked 08/19/2015}, the only solution is not making the entire contents of the editor editable, but implementing a layouting engine in JavaScript and enabling the editing mode only on parts of the layout.

%The issues arising with HTML editing APIs cannot be fixed. Many libraries find workarounds to treat them. CKEditor, TinyMCE, that framework. Google Docs finds another way and does not use HTML editing APIs.

%huge editor libraries, developed for 10 years trying to fix stuff
%libraries, not editors targeting inconstiencies
%they all can never know what's gonna happen
%medium

%\subsection{Clipboard} CKEditor ''fixes'' paste problems by implementing a custom fake context menu and opening a modal with some instruction that the user should press ctrl+v. This is a UX nightmare. Other editors like retractor (oder so) sanitize any change to the editors contents for the case of input by paste events.