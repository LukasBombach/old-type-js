%!TEX root = ../../thesis.tex
\chapter{Overview}
% \section{History and origins of HTML Editing APIs}

%HTML editing APIs have been poorly designed. Sections V to W discuss the origins and the reasons why HTML editing APIs have been standardized and are supported by all major browser. It can be seen, that they did not came to be because of their good design. In sections X and Y will discuss the design of these APIs.

% !!!

%The origins and the history of rich-text editing pose the question if the paradigms rich-text editing is based on have been thoroughly reviewed and if alternative ways for an implementation, possibly using hacks, could be considered.

HTML editing APIs are the standard and the recommended way by the W3C and the WHATWG for implementing rich-text editors on the web. However, its implementations across major web browsers are inconsistent, known to contain numerous bugs and have a limited and imprecise API.

This section will discuss the origins and adaption of HTML editing APIs as well as its advantages, disadvantages and alternatives.




However, they are full of bugs and have caused companies like Google, Microsoft and Apple and its downsides have been discussed by companies like Google and Medium.

creates unpredictable markup
its API is limited and imprecise