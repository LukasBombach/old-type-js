%!TEX root = ../thesis.tex
\thispagestyle{empty}
\noindent \textbf{Abstract}

\noindent Browsers do not offer native elements that allow for rich-text editing. There are third-party libraries that emulate these elements by utilizing the \texttt{contenteditable}-attribute. However, the API enabled by \texttt{contenteditable} is very limited and unstable. Bugs and unwanted behavior make it hard to use and can only be worked around, not fixed. By reviewing the API's history, it can be argued that its design has never been revisited only to ensure compatibility to current browsers. This thesis explains the API's downsides and demonstrates that rich-text editing can be achieved without requiring the \texttt{contenteditable}-attribute with the library ''Type'', thus solving many problems of contemporary third-party rich-text editors.