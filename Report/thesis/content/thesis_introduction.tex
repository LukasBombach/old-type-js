\section{Motivation}

Written text is the most important cultural tool to pass on knowledge from one generation to another. Designing texts, layouting, adding images and choosing text formattings is an important means to convey a message and to clarify ideas. A newspaper written as a single stream of words would not have served the purpose, that it has for hundreds of years.
With the computerization of printing, PCs have been the tool to generate and design texts for decades. There are software solutions for professional and personal use on Desktop PCs.
In recent years, many desktop applications have been migrated to browser-based solutions. This has many advantages. Applications can be maintained in a centralized manner and any computer using the software will be updated automatically. Browser-based applications can be accessed from anywhere in the world, without requiring to install further applications. Contents can be shared and edited collaboratively with others.

Still, rich-text editing, i.e. editing text that uses formattings and layouting, cannot be implemented easily in a browser. Browsers offer APIs for rich-text editing, but these APIs are very limited in its functionality, inconsistent across different browsers and known to contain numerous bugs.

This makes it hard for web developers to create rich-text editors. Usually, a third-party editor must be used and customized, which does not necessarily fit the specific needs of a project. The limited features of the browsers' rich-text APIs only allow for basic editors. A fully featured word-processing application like Google's document editor cannot be implemented with these APIs. For this reason Google omitted these APIs entirely. Unfortunately, there is no library and hardly any editor that implements rich-text editing without these APIs. Google did not publish their solution to the public domain.

The purpose of this thesis is to implement rich-text editing without using the browsers' rich-text editing APIs. This allows more features, a consistent behavior and avoids the bugs of these APIs. The implementation will be distributed as GUI-less software library with a high-level API, to enable web developers to implement rich-text editors specific to their needs, which is currently not possible.


\section{Terminology}

In web development, the term \textit{WYSIWYG} editor is commonly used to describe text-editors that allow formatting. WYSIWYG is an abbreviation for \textbf{W}hat \textbf{Y}ou \textbf{S}ee \textbf{I}s \textbf{W}hat \textbf{Y}ou \textbf{G}et and describes a text editor's capability to display formatted text as it is being edited. This stands out to plain-text editors that can neither display nor edit formattings. The term \textit{rich-text editor} has often been used for this feature and stands in better contrast to \textit{plain-text editor}. For this reason, the term \textit{rich-text editor} and \textit{rich-text editing} will be used in this thesis.

\section{Structure}

The first part of this thesis explains how rich-text editors are currently being implemented in browsers.

The second part discusses the problems with these approaches, possible alternatives as well as advantages and disadvantages of each approach.

Part three discusses techniques for an implementation of rich-text editing without rich-text editing APIs and part four discusses its implementation.

Part five gives an evaluation of this thesis.