%% This file should be \input (not \include 'd) from cover.tex.
Browsers do not offer native elements that allow for rich-text editing. There are third-party libraries that emulate these elements by utilizing the \texttt{contenteditable}-attribute. However, the API enabled by \texttt{contenteditable} is very limited and unstable. Bugs and unwanted behavior make it hard to use and can only be worked around, not fixed. By reviewing the APIs history, it can be argued that its design has never been revisited only to ensure compatibility to current browsers. This thesis explains the APIs downsides and demonstrates that rich-text editing can be achieved without requiring the \texttt{contenteditable}-attribute with library ''Type'', thus solving many problems that can be found contemporary third-party rich-text editors.


% Browsers do not offer native elements that allow for rich-text editing. There are third-party libraries that emulate these elements by utilizing the \texttt{contenteditable}-attribute. However, the API enabled by \texttt{contenteditable} is limited and unstable. Bugs and unwanted behavior can only be worked around and not fixed. The library ''Type'' demonstrates that rich-text editing can be achieved without requiring the \texttt{contenteditable} attribute, thus solving many problems contemporary third-party rich-text editor libraries have.