%% This is an example first chapter.  You should put chapter/appendix that you
%% write into a separate file, and add a line \include{yourfilename} to
%% main.tex, where `yourfilename.tex' is the name of the chapter/appendix file.
%% You can process specific files by typing their names in at the 
%% \files=
%% prompt when you run the file main.tex through LaTeX.
\chapter{Implementation}
\label{ch:implementation}

\section{JavaScript library development}

% There are many popular frameworks and libraries that provide a structure, rules and conventions for implementing websites and web applications. 

No IDEs, tools, not even conventions. Es gibt frameworks die eine architektur für webseiten und web-applikationen vorgeben, aber nichts dergleichen für bibliotheken.

Not for building:
Big JS libraries all do it differently.
Top 3 client side JavaScript repositories (stars) on github
https://github.com/search?l=JavaScript\&q=stars\%3A\%3E1\&s=stars\&type=Repositories
Angular.js: Grunt
d3 Makefile, also ein custom build script welche node packages aufruft
jQuery custom scripts, mit grunt und regex und so

Not for Architecture
Angular custom module system with own conventions
d3 mit nested objects (assoc arrays) und funktionen
jQuery mit .fn
ACE mit Klassen, daraus habe ich gelernt

ES5 oder ES6?


\section{Ich habe verwendet}

Gulp
requireJs
AMDClean
Uglify

JSLint - Douglas Crockford coding dogmatas / conventions
JSCS - JavaScript style guide checker

~~Livereload~~
PhantomJs
Mocha
Chai

Durch Require und AMDClean schöne arbeitsweise (am ende über bord geworfen) und kleine Dateigröße, wenig overhead.

Automatisierte Client side Tests mit PhantomJs und Mocha/Chai


\section{Coding conventions}

Habe mich größtenteils an Crockfordstyle orientiert, aber die Klassen anders geschrieben. Habe den Stil von ACE editor verwendet, denn der ist gut lesbar. Lesbarkeit war mir wichtiger als Crockford style. Für private Eigenschaften und Methoden habe ich die prefix convention verwedendet.
https://developer.mozilla.org/en-US/Add-ons/SDK/Guides/Contributor\_s\_Guide/Private\_Properties
Sie bewirkt keine echte accessibility restriction, aber es ist eine allgemein anerkannten convention und ist auch viel besser lesbar.

\section{Class pattern}

Ich habe mich für Klassen entschieden. Das hat folgende Vorteile:

* Klassen sind ein bewährtes Konzept um Code zu kapseln, logisch zu strukturieren und lesbar zu verwalten
* Durch prototypische Vererbung existiert die funktionalität von Klassen nur 1x im Browser 7 RAM
* Zudem gibt es Instanzvariablen, die für jede Type Instanz extra existieren und so mehrere Instanzen erlauben
* Die instanzvariablem sind meistens nur Pointer auf Instanzen anderer Klassen
* Das ganze ist dadurch sehr schlank

Eine alternative wäre das Modul-pattern, das ist aber schlechter zu lesen und bietet die oben genannten vorteile nur teilweise.

\section{Programmstruktur}

Modulbasiert?
Erweiterbarkeit
Es gibt ein Basisobjekt, das ist die Type "Klasse".
Darin werden dann die anderen Klassen geschrieben "Type.Caret", "Type.Selection", "Type.Range", ...
Das hat den Vorteil dass das ganze ge-name-spaced ist, so dass ich keine Konflikte mit Systemnamen habe (Range) (und auch nicht mit anderen Bibliotheken)
Effektiv gibt es eine (flache) Baumstruktur und so mit Ordnung. Für bestimmte Klassen, "Type.Event.Input", "Type.Input.Filter.X" geht es tiefer.
Der zweite Grund ist, dass ich somit alle Klassen die ich geschrieben habe für Entwickler sichtbar bereit stelle und nicht implizit und versteckt über irgend nen Quatsch.

Ursprünglich ein MVC konzept geplant mit einem Document Model und verschiedenen Renderern, aber über den haufen geworfen.

Ich werde jetzt die einzelnen Module erlären


\section{Type}

Die Type

\section{Range}
range



\section{Input}

%82.78\% of internet users support listening to and reading clipboard events. I can support 100\% PROBLEMS

There are many devices (hardware and virtual) a user can interact with native inputs:

\begin{enumerate} 
\item Keyboard (as hardware and as virtual keyboard)
\item Mouse
\item Touch
\item Game controller (on browsers imlpemented in game consoles)
\item Remote control (on Smart TV einvironments)
\end{enumerate}

When simulating a native input, in a best-case scenario, all these input methods should be accounted for. Fetching input needs to account for two scenarios: The user clicks / touches / or selects the input in any way and does so at any position inside the input. If the user touches / clicks / etc in the middle of the text, the caret should move to that position and typing must be enabled. In environments without hardware keyboards, the libarary must ensure that a virutal keyboards, possible native to the system, show up. Once the input is selected, text input must be fetched and written to the editor. There are various options to fetch user input, which will be discussed in the following paragraphs:

\paragraph{Events} One way to fetch user input is by listening to events. Text input can be read through \texttt{KeyboardEvent}s. Keyboard events will be triggered for virtual keyboards just as for hardware keyboards. When the user presses a key, the event can be stopped and the according characters can be inserted at the position of the caret. As a downside, listeners for keyboard events cannot be bound to a specific element that is not a native text input, that means keyboard events must be listened to on the \texttt{document} level. This not only has (minor) performance downsides but als requires more logic to decide whether a keyboard input should be processed and ultimatively stopped or ignored and allowed to bubble to other event listeners of a website. Furthermore, there can be edge cases, where even though a keyboard event should write contents to the editor, the event itself is supposed to trigger other methods that are not part of the editor. Keyboard events are supported by all major browsers across all devices.

To support clicking or touching inside the editor's contents \texttt{MouseEvent}s and \texttt{TouchEvent}s can be used. Mouse events are supported on all major desktop browsers and all mobile browsers support touch events. Both event types support reading the coordinates indicating where the click or touch has been performed.

Although some smart TVs offer keyboards, mice, pointers similar to Nintendo's Wii remote, input via smartphone apps and many others, button-based remote controls are offered with almost any smart TV and remain a edge case for interacting with a text editor. In such an enviroment, users commonly switch between elements by selecting focusable elements with a directional pad. Using only events would not account for this case since there would be no focusable element representing the editor. Recent browsers on Samsung's and LG's smart TVs are based on WebKit\footnote{\url{http://www.samsungdforum.com/Devtools/Sdkdownload}, last checked on 07/22/2015} while Sony's TVs use Opera. Before 2012 Samsung's browser was based on Gecko. All of these browsers and browser engines supprt keyboard events to fetch input.

\paragraph{Clipboard} Another problem with relying entirely on events is the lack of native clipboard capabilities. Unless a native text input (including elememts with enabled editing mode) is focused, shortcut keys for pasting will not trigger a paste event and the mouse's context menu will not offer an option for pasting. Recent versions of Chrome, Opera and Android Browser\footnote{\url{http://caniuse.com/\#feat=clipboard}, last checked on 07/22/2015} allow triggering arbitrary paste events thereby read the cliboard contents. With this, shortcuts could be enabled with JavaScript and instead of the native context menu, a customly build context menu using HTML could be shown that allows the user to paste, but this only works on elements in editing mode and only in these three browsers.

\paragraph{Hidden native input fields} Apart from rich-text editors there are also third-party JavaScript libraries that allow embedding code editors with syntax highlighting in a website. The ''Ajax.org Cloud9 Editor (Ace)'' and ''CodeMirror'' editor are amongst the most well-known choices. Both editors keep an internal representation of the plain-text (i.e. non-highlighted) contents of the editor, parse it and display a highlighted version using HTML in a designated \texttt{div}. Both editors use a hidden \texttt{textarea} in which the user enters his or her input. The input will be read from the \texttt{textarea} and

% which is presented to the user as the input field.

\section{Caret}
caret
\paragraph{BiDi support}
\paragraph{IME} 
http://marijnhaverbeke.nl/blog/browser-input-reading.html 
https://en.wikipedia.org/wiki/Input\_method


\section{Selection}
Fake damit copy \& past nativ funktioniert
Obwohl eine selection mehrere Ranges haben kann, wird das von keinem Browser unterstützt http://stackoverflow.com/questions/4777860/highlight-select-multiple-divs-with-ranges-w-contenteditable/4780571\#4780571
Habe versucht mit die Darstellung über getClientRects zu emulierten, aber das ist buggy siehe https://github.com/edg2s/rangefix/blob/master/rangefix.js


\section{Selection Overlay}
overlay

\section{Formatting}
formatting
\section{Change Listener}
change
\section{Contents}
contents
\section{Development}
developmnent
\section{Dom Utilities}
dom util
\section{Dom Walker}
dom walker. Used SO OFTEN in the project. Goal is to have a really simple api new DomWalker('text') by having lots of pre-defined magick in it.
\section{Environment}
env

\section{Core Api}

Api als eigenes Modul, separation of concerns, gute saubere code base und eigene saubere code base für die API. Laesst freiheiten in der Softwarearchitektur - lost coupling, API ein eigenes aenderbares gekapseltes modul:

\section{Event Api}
ev api

\section{Events}
\paragraph{Input}
input event only event required so far

\section{Input Pipeline}
pipeline idea
rules for behaviours (lists, headlines, enter, spaces...)
Ursprünglich so gedacht: Was gehört zu einem editor X, Y und Behaviour. Für Behaviour aber das input pipeline system gefunden, welches das ganze schön abbildet/löst.
Noch mal skimmen und sehen ob da was dabei ist, besonders "Detour": http://marijnhaverbeke.nl/blog/browser-input-reading.html

\paragraph{Pasting} Kontrolle durch die Implementierung mit einem versteckten Eingabefeld. Erst mal gibt es ein reliable Paste event, was es in keinem Editor basierend auf contentEditable gibt. 2. Hat man volle Kontrolle über den Inhalt, was gepastet wird und was nicht. Und das ganze mit einer einfachen API. Pasting sollte internes event ausloesen.


\paragraph{Caret}
caret
\paragraph{Command}
command
\paragraph{Headlines}
head lines
\paragraph{Line Breaks}
line breaks
\paragraph{Lists}
lists
\paragraph{Remove}
remove
\paragraph{Spaces}
spaces

\section{Plugin Api}
plugin api
\section{Settings}
settings
\section{Text Walker}
text walker
\section{Utilities}
util

\section{Extensions}

\paragraph{Plugin API} ...and plugin cache

\paragraph{Many options for programmers}

\begin{lstlisting}[language=JavaScript, caption=Example calls to format text, label=lst:format-examples]
// Adding contructors
Type.fromEtherpad = function() { /* Implementation */ };

// Events to load plugins on instantiation
Type.on('ready', function(type) { /* Implementation */ });

// Lazy loading using plugin cache
Type.fn.myPluginMethod = function (tag, params) {
  var cmd = this.plugin('cmd', /* Plugin instantiation */ );
  /* Execute myPluginMethod */
};

// Static function calls just like jQuery
Type.fn.cmd = function (tag, params) { /* Implementation */ };
\end{lstlisting}

Erstellung von Erweiterungen und Plugins soll so einfach und frei wie möglich sein. Es ist die Verantwortung der Programmierer diese Freiheit so zu nutzen, dass sie trotzdem sauber arbeiten. jQuery zeigt, dass Freiheit funktioneren kann und nicht zu schlechtem nutzen führt. Schlechte Plugins sterben durch die Dynamik des ''Marktes'' aus. Die Vorteile ueberwiegen: Beim CKEditor ist es aufwändig (viel Code um sich in den Editor zu integrieren) und schwierig (Lernkurve) Erweiterungen zu schreiben. Der CKEditor ist aber auch ein Produkt und kein Framework. Ein Framework sollte Möglichkeiten geben und einfach (und painless) sein. jQuery hat für sich Konventionen erzeugt und Möglichkeiten gegen, die Programmierer effektiv arbeiten lassen können. Und jQuery zeigt: Es funktioniert.

\paragraph{Etherpad} Collaboration with etherpad ist wirklich einfach zu machen mit meinem coolen editor. Mein Editor ist ganz ganz toll dass das so einfach geht, ja!
Markdown wohl eher als Ausblick.
