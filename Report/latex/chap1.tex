%% This is an example first chapter.  You should put chapter/appendix that you
%% write into a separate file, and add a line \include{yourfilename} to
%% main.tex, where `yourfilename.tex' is the name of the chapter/appendix file.
%% You can process specific files by typing their names in at the 
%% \files=
%% prompt when you run the file main.tex through LaTeX.
\chapter{DOM manipulation}

In October 1998 the World Wide Web Consortium (W3C) published the ''Document Object Model (DOM) Level 1 Specification''. This specification includes an API on how to alter DOM nodes and the document's tree\footnote{\url{http://www.w3.org/TR/REC-DOM-Level-1/level-one-core.html}, last checked on 07/10/2015}. It provided a standardized way for changing a website's contents. With the implementations of Netscape's JavaScript and Microsoft's JScript this API has been made accessible to web developers.

\section{HTML Editing APIs}

In July 2000, with the release of Internet Explorer 5.5, Microsoft introduced the contentEditable and designMode IDL attributes along with the contenteditable content attribute\footnote{\url{https://msdn.microsoft.com/en-us/library/ms533720(v=vs.85).aspx}, last checked on 07/10/2015}\footnote{\url{https://msdn.microsoft.com/en-us/library/ms537837(VS.85).aspx}, last checked on 07/10/2015}. These attributes were not standardized and not part of the W3C DOM specifications. 

% Please add the following required packages to your document preamble:
% \usepackage{graphicx}
\begin{table}[]
\centering
\resizebox{\textwidth}{!}{%
\begin{tabular}{llll}
\hline
Attribute       & Type              & Can be set to         & Possible values                     \\ \hline
designMode      & IDL attribute     & Document              & "on", "off"                         \\
contentEditable & IDL attribute     & Specific HTMLElements & boolean, "true", "false", "inherit" \\
contenteditable & content attribute & Specific HTMLElements & empty string, "true", "false"       \\ \hline
\end{tabular}
}
\caption{Editing API attributes}
\label{editing-api-attributes}
\end{table}

By setting \texttt{contenteditable} or \texttt{contentEditable} to ''true'' or \texttt{designMode} to ''on'', Internet Explorer switches the affected elements and their children to an editing mode. In editing mode it is possible to

\begin{enumerate} \item Let the user interactively click on and type inside text elements \item Execute ''commands'' via JScript and JavaScript\end{enumerate}

Fetching user inputs (clicking on elements, accepting keyboard input and modifying text nodes) is handled entirely by the browser. No further scripting is necessary other than setting the mentioned attributes on elements. This behavior is inherited by child elements.

In editing mode, calling the method \texttt{document.execCommand} will format the currently selected text. Calling \texttt{document.execCommand('bold', false, null)} will wrap the currently selected text in \texttt{<b>} tags. \texttt{document.execCommand('createLink', false, 'http://google.com/')} will wrap the selected text in a link to google.com. However, this command will be ignored, if the current selection is not contained by an element in editing mode.

While \texttt{designMode} can only be applied to the entire document, \texttt{contentEditable} and \texttt{contenteditable}  attributes can be applied to a subset of HTML elements as described on Microsoft's Developer Network (MSDN) online documentation\footnote{\url{https://msdn.microsoft.com/en-us/library/ms537837(VS.85).aspx}, last checked on 07/10/2015}.

With the release of Internet Explorer 5.5 and the introduction of editing capabilities, Microsoft released a sparse documentation\footnote{\url{https://msdn.microsoft.com/en-us/library/ms537837(VS.85).aspx}, last checked on 07/10/2015} describing only the availability and the before-mentioned element restrictions of these attributes. 

According to Mark Pilgrim, author of the ''Dive into'' book series and contributor to the the Web Hypertext Application Technology Working Group (WHATWG), Microsoft did not state a specific purpose for its editing API, but, its first use-case has been rich-text editing\footnote{\url{https://blog.whatwg.org/the-road-to-html-5-contenteditable}, last checked on 07/10/2015}.

In March 2003, the Mozilla Foundation introduced an implementation of Microsoft's designMode, named Midas, for their release of Mozilla 1.3. Mozilla names this ''rich-text editing support'' on the Mozilla Developer Network (MDN)\footnote{\url{https://developer.mozilla.org/en/docs/Rich-Text\_Editing\_in\_Mozilla}, last checked on 07/10/2015}. In June 2008, Mozilla added support for contentEditable IDL and contenteditable content attributes with Firefox 3. 

Mozilla's editing API resembles the API implemented for Internet Explorer, however, there are still differences (compare \footnote{\url{https://msdn.microsoft.com/en-us/library/hh772123(v=vs.85).aspx}, last checked on 07/10/2015}\footnote{\url{https://developer.mozilla.org/en-US/docs/Midas}, last checked on 07/10/2015}). Most notably, Microsoft and Mozilla differ in the commands provided to pass to document.execCommand\footnote{\url{https://developer.mozilla.org/en-US/docs/Midas}, last checked on 07/10/2015}\footnote{\url{https://msdn.microsoft.com/en-us/library/ms533049(v=vs.85).aspx}, last checked on 07/10/2015} and the markup generated by invoking commands\footnote{\url{https://developer.mozilla.org/en/docs/Rich-Text\_Editing\_in\_Mozilla#Internet\_Explorer\_Differences}, last checked on 07/10/2015}. In fact, Mozilla only provides commands dedicated to text editing while Microsoft offers a way to access lower-level browser components (like the browser's cache) using execCommand. This may show, that even though rich-text editing was its first use case and Mozilla implemented it naming it that, this editing API was not originally intended to be used as such.

In March 2008, Apple released Safari 3.1\footnote{\url{https://www.apple.com/pr/library/2008/03/18Apple-Releases-Safari-3-1.html}, last checked on 07/10/2015} including full support for contentEditable and designMode\footnote{\url{http://caniuse.com/#feat=contenteditable}, last checked on 07/10/2015}, followed by Opera Software in June 2006\footnote{\url{http://www.opera.com/docs/changelogs/windows/}, last checked on 07/10/2015} providing full support in Opera 9\footnote{\url{http://www.opera.com/docs/changelogs/windows/900/}, last checked on 07/10/2015}. MDN lists full support in Google Chrome since version 4\footnote{\url{https://developer.mozilla.org/en-US/docs/Web/Guide/HTML/Content\_Editable}, last checked on 07/10/2015}, released in January 2010\footnote{\url{http://googlechromereleases.blogspot.de/2010/01/stable-channel-update\_25.html}, last checked on 07/10/2015}.

Around the year 2003[\textit{MeineTabelle}] the first JavaScript libraries emerged that made use of Microsoft's and Mozilla's editing mode to offer rich-text editing in the browser. Usually these libraries were released as user interface components (text fields) with inherent rich-text functionality and were only partly customizable.

In May 2003 and March 2004 versions 1.0 of ''FCKEditor'' and ''TinyMCE'' have been released as open source projects. These projects are still being maintained and remain among the most popular choices for incorporating rich-text editing in web-based projects. \textit{// Technik, wie diese Editoren funktionieren erklären.}
