%% This is an example first chapter.  You should put chapter/appendix that you
%% write into a separate file, and add a line \include{yourfilename} to
%% main.tex, where `yourfilename.tex' is the name of the chapter/appendix file.
%% You can process specific files by typing their names in at the 
%% \files=
%% prompt when you run the file main.tex through LaTeX.
\chapter{Concept}


\section{Conformity with HTML Editing APIs}
Wie sehr passt meine library zu dein HTML Editing APIs
Was definieren die?
Was muss ich conformen, welche Freiheiten habe ich?
https://dvcs.w3.org/hg/editing/raw-file/tip/editing.html

\section{Goals}

\paragraph{Minimze interaction with the DOM} DOM operations are slow and should be avoided.

\paragraph{Minimze interaction with unstable APIs} Some APIs like the \texttt{Range} or \texttt{Selection} are prone to numerous bugs. To maximize stability, these APIs should be avoided when possible unless doing so has any downsides (like lower performance).

\paragraph{Leave as much implementation to the browser as possible} "Easy to make it fast -- The browser (not the app) handles the most computationally intensive task: text layout. Since layout is a core component of browser functionality, you can trust that layout performance has already been heavily optimized." http://googledrive.blogspot.fr/2010/05/whats-different-about-new-google-docs.html

\paragraph{Markup} Our editor should be a good citizen in this ecosystem. That means we ought to produce HTML that’s easy to read and understand. And on the flip side, we need to be aware that our editor has to deal with pasted content that can’t possibly be created in our editor. https://medium.com/medium-eng/why-contenteditable-is-terrible-122d8a40e480


\section{MVC}

Document model -> no