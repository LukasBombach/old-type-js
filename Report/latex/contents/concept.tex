%% This is an example first chapter.  You should put chapter/appendix that you
%% write into a separate file, and add a line \include{yourfilename} to
%% main.tex, where `yourfilename.tex' is the name of the chapter/appendix file.
%% You can process specific files by typing their names in at the 
%% \files=
%% prompt when you run the file main.tex through LaTeX.
\chapter{Concept}

\section{Approaches for enabling rich-text editing}
%When not using HTML editing APIs, the components and the bahavior of native text inputs must be imitated. There are various approaches to this.

%\paragraph{Overlaying an element in editing mode} One way to 

\paragraph{Enabling editing mode without using its API} One way to enable editing but avoiding many bugs and browser inonstistencies, is to enable the editing mode on an element, but avoid using \texttt{execCommand} to format the text. The latter could be implemented using the DOM core APIs. All text editing functions, i.e. the caret, text input, mouse interaction and clipboard capabilites would be taken care of by the browser.

This would solve the problem of buggy and inconsistent \texttt{execCommand} implementations but not the problems that arise with different browser behavior on the user's text input---for instance when entering a line break. If the markup is customly generated with JavaScript, the input may not function with it and / or break customly generated elements. It could be the source to many bugs, have the development of the library be dependent on browser development and ultimatively restrict the editors capabilites.


%\section{Native text input imitation}

\section{Conformity with HTML Editing APIs}
Wie sehr passt meine library zu dein HTML Editing APIs
Was definieren die?
Was muss ich conformen, welche Freiheiten habe ich?
https://dvcs.w3.org/hg/editing/raw-file/tip/editing.html

\section{Leave as much implementation to the browser as possible} 

"Easy to make it fast -- The browser (not the app) handles the most computationally intensive task: text layout. Since layout is a core component of browser functionality, you can trust that layout performance has already been heavily optimized." http://googledrive.blogspot.fr/2010/05/whats-different-about-new-google-docs.html

\section{Markup} 

Our editor should be a good citizen in this ecosystem. That means we ought to produce HTML that’s easy to read and understand. And on the flip side, we need to be aware that our editor has to deal with pasted content that can’t possibly be created in our editor. https://medium.com/medium-eng/why-contenteditable-is-terrible-122d8a40e480

Also FirePad and Google generate unusable markup. we're gonna generate semantic markup, simple and clean.

%\section{Partially using HTML editing APIs}

%That would be one way \textbf{note: try to show all the options for implementation for WeWu}

\section{MVC}

Document model -> no


\section{Stability and performance}

\paragraph{Cache} For traversing the text, for example when the caret moves, the text will need to be measured. All measurements can be stored to a cache to only perform the same measurement operations once. On the other hand, the library should be a ''good citizen'' on a website, which means it should be as unobtrusive and leave the developers as much freedom as possible. The library will essentially modify parts of the DOM that act as the editor's contents. It should be agnostic to the editor's contents at all times to give other developers the freedom to change the contents in any way needed without breaking the editor. A cache must account for external changes. The DOM3 Events specification\footnote{\url{http://www.w3.org/TR/DOM-Level-3-Events/}, last checked on 07/21/2015} offers \texttt{MutationObserver}s to check for DOM changes. This feature is not supported by Internet Explorer version 10 or less\footnote{\url{http://caniuse.com/\#search=mutation}, last checked on 07/21/2015}. Internet Explorer 9 and 10 offer an implementation for \texttt{MutationEvent}s\footnote{\url{http://help.dottoro.com/ljfvvdnm.php\#additionalEvents }, last checked on 07/21/2015}. The W3C states that ''The MutationEvent interface [...] has not yet been completely and interoperably implemented across user agents. In addition, there have been critiques that the interface, as designed, introduces a performance and implementation challenge.''\footnote{\url{http://www.w3.org/TR/DOM-Level-3-Events/\#legacy-mutationevent-events}, last checked on 07/21/2015}. Apart from that, the benefits of a cache may not signifcantly increase the library's performance. The actions that can be supported by a cache, most importantly moving the caret in the text, are not very complex and do not noticably afflict the CPU. Implementing an editor that is stateless in regards of its contents can also improve stability.

% dom operations could also be cached

\paragraph{Minimze interaction with the DOM} DOM operations are slow and should be avoided. DOM operations can be ''collected'' and only executed when necessary etc pp like React.

\paragraph{Minimze interaction with unstable APIs} Some APIs like the \texttt{Range} or \texttt{Selection} are prone to numerous bugs. To maximize stability, these APIs should be avoided when possible unless doing so has any downsides (like lower performance).

