%% This is an example first chapter.  You should put chapter/appendix that you
%% write into a separate file, and add a line \include{yourfilename} to
%% main.tex, where `yourfilename.tex' is the name of the chapter/appendix file.
%% You can process specific files by typing their names in at the 
%% \files=
%% prompt when you run the file main.tex through LaTeX.

\chapter{Text editing in desktop environments}
%\chapter{Word processing on desktop PCs}

\section{Basics of plain-text editing} % word-processing

caret
selection
input

\section{Basics Rich-text editing} % word-processing

document tree
formatting algorithms

% we will see that contenteditable handles all this, but in my solution this all needs to be coded again
% I need to discuss that the benefits are bigger than having to do this

\section{Libraries for desktop environments}

It is no longer needed to implement basic rich-text editing components from the ground up. Rich-text editing has become a standard and most modern Frameworks, system APIs or GUI libraries come with built-in capabilites. Table \ref{table:rich-text-components-desktop} lists rich-text text components for popular languages and frameworks.

\begin{table}[]
\centering
\begin{tabular}{ll}
\hline
Environment & Component \\ \hline
Java (Swing) & JTextPane / JEditorPane \\
MFC & CRichEditCtrl \\
Windows Forms / .NET & RichTextBox \\
Cocoa & NSTextView \\
Python & Tkinter Text \\
Qt & QTextDocument \\ \hline
\end{tabular}
\caption{Rich-text components in desktop environments}
\label{table:rich-text-components-desktop}
\end{table}

% \paragraph{MFC}
% https://msdn.microsoft.com/en-us/library/68730ktd.aspx

% \paragraph{.NET}
% https://msdn.microsoft.com/en-us/library/system.windows.forms.richtextbox(v=vs.110).aspx

% \paragraph{Cocoa}
% https://developer.apple.com/library/mac/documentation/Cocoa/Reference/ApplicationKit/Classes/NSTextView\_Class/index.html

% \paragraph{Python}
% http://infohost.nmt.edu/tcc/help/pubs/tkinter/web/text.html

% \paragraph{Qt}
% http://doc.qt.io/qt-5.5/richtext.html