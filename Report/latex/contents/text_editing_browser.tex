%% This is an example first chapter.  You should put chapter/appendix that you
%% write into a separate file, and add a line \include{yourfilename} to
%% main.tex, where `yourfilename.tex' is the name of the chapter/appendix file.
%% You can process specific files by typing their names in at the 
%% \files=
%% prompt when you run the file main.tex through LaTeX.
\chapter{Text editing in browser environments}

% \section{Differing technical requirements in browser environments}
% Browser environment differs from desktop environments. 
% Memory management, garbage collection, none of that is necessary
% The basis for web projects is the DOM and the data and procedures written in JavaScript
% Hacks are normal

\section{Plain-text inputs}

Text input components for browsers have been introduced with the specification of HTML 2.0\footnote{\url{https://tools.ietf.org/html/rfc1866}, last checked on 07/15/2015}. The components proposed include inputs for single line (written as \texttt{<input type=''text'' />}) and multiline texts (written as \texttt{<textarea></textarea>}). These inputs allow writing plain-text only.


\section{Rich-text editing}

Major browsers, i.e. any browser with a market share above 0.5\%\footnote{\url{http://gs.statcounter.com/#all-browser-ww-monthly-201406-201506-bar}, last checked on 07/15/2015}, do not offer native input fields that allow rich-text editing. Neither the W3C's HTML5 and HTML5.1 specifications nor the WHATWG HTML specification recommend such elements. However, by being able to display HTML, browsers effectively are rich-text viewers. By the early 2000s, the first JavaScript libraries emerged, that allowed users to interactively change (parts of) the HTML of a website, to make rich-text editing in the browser possible. The techniques used will be discussed in section~\ref{sec:html-editing-apis} to section~\ref{sec:useage-of-html-editing-apis}.

\subsection{HTML Editing APIs}
\label{sec:html-editing-apis}

In July 2000, with the release of Internet Explorer 5.5, Microsoft introduced the IDL attributes \texttt{contentEditable} and \texttt{designMode} along with the content attribute \texttt{contenteditable}\footnote{\url{https://msdn.microsoft.com/en-us/library/ms533720(v=vs.85).aspx}, last checked on 07/10/2015}\footnote{\url{https://msdn.microsoft.com/en-us/library/ms537837(VS.85).aspx}, last checked on 07/10/2015}. These attributes were not part of the W3C's HTML 4.01 specifications\footnote{\url{http://www.w3.org/TR/html401/}, last checked on 07/14/2015} or the ISO/IEC 15445:2000\footnote{\url{http://www.iso.org/iso/iso\_catalogue/catalogue\_tc/catalogue\_detail.htm?csnumber=27688}, last checked on 07/14/2015}, the defining standards of that time. Table \ref{table:editing-api-attributes} lists these attributes and possible values.

% Please add the following required packages to your document preamble:
% \usepackage{graphicx}
\begin{table}[]
\centering
\resizebox{\textwidth}{!}{%
\begin{tabular}{llll}
\hline
Attribute       & Type              & Can be set to         & Possible values                     \\ \hline
designMode      & IDL attribute     & Document              & "on", "off"                         \\
contentEditable & IDL attribute     & Specific HTMLElements & boolean, "true", "false", "inherit" \\
contenteditable & content attribute & Specific HTMLElements & empty string, "true", "false"       \\ \hline
\end{tabular}
}
\caption{Editing API attributes}
\label{table:editing-api-attributes}
\end{table}

By setting \texttt{contenteditable} or \texttt{contentEditable} to ''true'' or \texttt{designMode} to ''on'', Internet Explorer 5.5 switches the affected elements and their children to an editing mode. This mode makes it possible to

\begin{enumerate} \item Let the user interactively click on and type inside text elements \item Execute ''commands'' via JScript and JavaScript\end{enumerate}

Fetching user inputs (setting the caret by clicking on elements, accepting keyboard input and modifying text nodes) is handled entirely by the browser. No further scripting is necessary. This behavior is inherited by child elements.

The editing mode enables an API, that can be called globally on the document object. Table \ref{table:editing-mode-api} lists the full API enabled this way.

In editing mode, calling the method \texttt{document.execCommand} will format the currently selected text. Calling \texttt{document.execCommand('bold', false, null)} will wrap the currently selected text in \texttt{<b>} tags. \texttt{document.execCommand('createLink', false, 'http://google.com/')} will wrap the selected text in a link to google.com. However, this command will be ignored, if the current selection is not contained by an element in editing mode.

While \texttt{designMode} can only be applied to the entire document, \texttt{contentEditable} and \texttt{contenteditable}  attributes can be applied to a subset of HTML elements as described on Microsoft's Developer Network (MSDN) online documentation\footnote{\url{https://msdn.microsoft.com/en-us/library/ms537837(VS.85).aspx}, last checked on 07/10/2015}.

With the release of Internet Explorer 5.5 and the introduction of editing capabilities, Microsoft released a sparse documentation\footnote{\url{https://msdn.microsoft.com/en-us/library/ms537837(VS.85).aspx}, last checked on 07/10/2015} describing only the availability and the before-mentioned element restrictions of these attributes. 

According to Mark Pilgrim, author of the ''Dive into'' book series and contributor to the the Web Hypertext Application Technology Working Group (WHATWG), Microsoft did not state a specific purpose for its editing API, but, its first use-case has been rich-text editing\footnote{\url{https://blog.whatwg.org/the-road-to-html-5-contenteditable}, last checked on 07/10/2015}.

In March 2003, the Mozilla Foundation introduced an implementation of Microsoft's designMode, named Midas, for their release of Mozilla 1.3. Mozilla names this ''rich-text editing support'' on the Mozilla Developer Network (MDN)\footnote{\url{https://developer.mozilla.org/en/docs/Rich-Text\_Editing\_in\_Mozilla}, last checked on 07/10/2015}. In June 2008, Mozilla added support for contentEditable IDL and contenteditable content attributes with Firefox 3. 

Mozilla's editing API resembles the API implemented for Internet Explorer, however, there are still differences (compare \footnote{\url{https://msdn.microsoft.com/en-us/library/hh772123(v=vs.85).aspx}, last checked on 07/10/2015}\footnote{\url{https://developer.mozilla.org/en-US/docs/Midas}, last checked on 07/10/2015}). Most notably, Microsoft and Mozilla differ in the commands provided to pass to document.execCommand\footnote{\url{https://developer.mozilla.org/en-US/docs/Midas}, last checked on 07/10/2015}\footnote{\url{https://msdn.microsoft.com/en-us/library/ms533049(v=vs.85).aspx}, last checked on 07/10/2015} and the markup generated by invoking commands\footnote{\url{https://developer.mozilla.org/en/docs/Rich-Text\_Editing\_in\_Mozilla#Internet\_Explorer\_Differences}, last checked on 07/10/2015}. In fact, Mozilla only provides commands dedicated to text editing while Microsoft offers a way to access lower-level browser components (like the browser's cache) using execCommand. This may show, that even though rich-text editing was its first use case and Mozilla implemented it naming it that, this editing API was not originally intended to be used as such.

In March 2008, Apple released Safari 3.1\footnote{\url{https://www.apple.com/pr/library/2008/03/18Apple-Releases-Safari-3-1.html}, last checked on 07/10/2015} including full support for contentEditable and designMode\footnote{\url{http://caniuse.com/#feat=contenteditable}, last checked on 07/10/2015}, followed by Opera Software in June 2006\footnote{\url{http://www.opera.com/docs/changelogs/windows/}, last checked on 07/10/2015} providing full support in Opera 9\footnote{\url{http://www.opera.com/docs/changelogs/windows/900/}, last checked on 07/10/2015}. MDN lists full support in Google Chrome since version 4\footnote{\url{https://developer.mozilla.org/en-US/docs/Web/Guide/HTML/Content\_Editable}, last checked on 07/10/2015}, released in January 2010\footnote{\url{http://googlechromereleases.blogspot.de/2010/01/stable-channel-update\_25.html}, last checked on 07/10/2015}.

Around the year 2003[\textit{MeineTabelle}] the first JavaScript libraries emerged that made use of Microsoft's and Mozilla's editing mode to offer rich-text editing in the browser. Usually these libraries were released as user interface components (text fields) with inherent rich-text functionality and were only partly customizable.

In May 2003 and March 2004 versions 1.0 of ''FCKEditor'' and ''TinyMCE'' have been released as open source projects. These projects are still being maintained and remain among the most popular choices for incorporating rich-text editing in web-based projects. \textit{// Technik, wie diese Editoren funktionieren erklären.}

Seeing the history of editing APIs, it is understandable how this has become the standard for rich-text editing. However, with its introduction in Internet Explorer 5.5, it has not been stated that the \texttt{designMode} end \texttt{contentEditable} attributes have been intended to enable rich-text editing. Sections X and Y will discuss the advantages and disadvantages of these APIs.

\subsection{Usage of HTML Editing APIs}
\label{sec:useage-of-html-editing-apis}

How Js Libraries work. Maybe how only a few work. StackOverlfow quote, buglists, Medium post, other stuff to find.

% Rich-text editing in browsers is only possible though JavaScript. Essentially, libraries enabling rich-text editing display a nested webpage through an iFrame and let the user modify its contents to emulate a rich-text input. Commonly, modification is realized though the browser's so-called ''HTML Editing APIs'', which will be discussed in Section XXX (HTML Editing APIs).

\subsection{Advantages of HTML Editing APIs}

Higlevel API
Wenig Aufwand
discussion of WHATWG

\subsection{Disadvantages of HTML Editing APIs}

bugs
limitations
No specifications on what markup to generate (mozilla != ie, mdn has links for that)

\subsection{DOM manipulation without Editing APIs}

In October 1998 the World Wide Web Consortium (W3C) published the ''Document Object Model (DOM) Level 1 Specification''. This specification includes an API on how to alter DOM nodes and the document's tree\footnote{\url{http://www.w3.org/TR/REC-DOM-Level-1/level-one-core.html}, last checked on 07/10/2015}. It provided a standardized way for changing a website's contents. With the implementations of Netscape's JavaScript and Microsoft's JScript this API has been made accessible to web developers.

\subsection{Rich-text editing without Editing APIs}
