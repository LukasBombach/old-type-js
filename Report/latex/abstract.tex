% $Log: abstract.tex,v $
% Revision 1.1  93/05/14  14:56:25  starflt
% Initial revision
% 
% Revision 1.1  90/05/04  10:41:01  lwvanels
% Initial revision
% 
%
%% The text of your abstract and nothing else (other than comments) goes here.
%% It will be single-spaced and the rest of the text that is supposed to go on
%% the abstract page will be generated by the abstractpage environment.  This
%% file should be \input (not \include 'd) from cover.tex.
Browsers do not offer native elements that allow for rich-text editing. There are third-party libraries that emulate these elements by utilizing the \texttt{contenteditable}-attribute. However, the API enabled by \texttt{contenteditable} is limited and unstable. Bugs and unwanted behavior can only be worked around and not fixed. The library ''Type'' demonstrates that rich-text editing can be achieved without requiring the \texttt{contenteditable} attribute, thus solving many problems contemporary third-party rich-text editor libraries have.
