\chapter{Introduction}
\section{Motivation}

Rich-text editors are commonly used by many on a daily basis. Often, this happens knowingly, for instance in an office suite, when users wilfully format text. But often, rich-text editors are being used without notice. For instance when writing e-mails, entering a URL inserts a link automatically in many popular e-mail-applications. Also, many applications, like note-taking apps, offer rich-text capabilites that go unnoticed. Many users do not know the difference between rich-text and plain-text writing. Rich-text editing has become a de-facto standard, that to many users is \textit{just there}. Even many developers do not realise that formatting text is a feature that needs special implementation, much more complex than plain-text editing.

While there are APIs for creating rich-text input controls in many desktop programming environments, web-browsers do not offer native rich-text inputs. However, third-party JavaScript libraries fill the gap and enable developers to include rich-text editors in web-based projects.

Still, these libraries come with critical downsides


\section{Disclaimer}


\section{Structure}
