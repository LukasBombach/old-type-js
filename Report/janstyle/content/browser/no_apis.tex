%!TEX root = ../../thesis.tex
\section{Rich-text editing without editing APIs}
%% \section{DOM manipulation without Editing APIs}

HTML editing APIs are the recommended way for implementing a web-based rich-text editor. There is no native text input that can display formatted text. The only way to natively display rich-text on a website is through the Document Object Model (DOM). Editors based on HTML editing APIs utilize the DOM to display their rich-text contents too. Only the editing (of the DOM), commonly phrased ''DOM manipulation'', is implemented with HTML editing APIs.

\subsection{Manipulation via the DOM APIs} Manipulating the DOM has been possible since the first implementations of JavaScript and JScript. It has been standardised in 1998 with the W3C's ''Document Object Model (DOM) Level 1 Specification'' as part of the ''Document Object Model (Core) Level 1''\footnote{\url{http://www.w3.org/TR/REC-DOM-Level-1/level-one-core.html}, last checked on 07/10/2015}.

The DOM and its API is the recommended\footnote{recommended by the W3C and WHATWG} way to change a website's contents and---apart from HTML editing APIs---the only possibility \textit{natively} implemented in any major browser. Popular libraries like jQuery, React or AngularJS are based on it. The API has been developed for 17 years and proven to be stable across browsers. Any DOM manipulation that can be achieved with HTML editing APIs can also be achieved using this API.

%% %%%%%%%%%%%%
%% contenteditable is also DOM API (but not core API?)
%% %%%%%%%%%%%%

\begin{algorithm}
\caption{Simplified text formatting pseudocode}
\label{alg:format_pseudocode}
\begin{algorithmic}[1]
\State Split text nodes at beginning and end of selection
\State Create a new tag before the beginning text node of the selection
\State Loop over all nodes inside selection
\State Move each node inside new tag
\end{algorithmic}
\end{algorithm}

% \State Move beginning text node from selection inside new tag
% \State Move all nodes until end text node inside new tag
% \State Move end text node inside new tag

Algorithm~\ref{alg:format_pseudocode} demonstrates a simplified procedure to wrap a text selection in a tag. To implement the \texttt{bold} commmand of \texttt{execCommand}, this procedure can be implemented using the \texttt{strong} tag. The text selection can be read with the browser's selection API\footnote{\url{https://developer.mozilla.org/en-US/docs/Web/API/Selection}, last checked on 07/18/2015}\footnote{Internet Explorer prior version 9 uses a non-standard API \url{https://msdn.microsoft.com/en-us/library/ms535869(v=VS.85).aspx}, last checked on 07/18/2015}.

% To implement formatting commands, the currently selected text can be wrapped in newly created tags, depending on what formatting shall be achieved. For instance, to implement the \texttt{bold} commmand of \texttt{execCommand}, the selected text can be read with the browser's selection API\footnote{\url{https://developer.mozilla.org/en-US/docs/Web/API/Selection}, last checked on 07/18/2015}\footnote{Internet Explorer prior version 9 uses a non-standard API \url{https://msdn.microsoft.com/en-us/library/ms535869(v=VS.85).aspx}, last checked on 07/18/2015} and wrapped in \texttt{strong} tags using basic DOM Level 1 methods. 

\begin{algorithm}
\caption{Simplified element insertion pseudocode}
\label{alg:insert_pseudocode}
\begin{algorithmic}[1]
\State If\{Selection is not collapsed\}\{
\State Split text nodes at beginning and end of selection
\State Loop over all nodes inside selection
\State Remove each node
\State Collapse selection\}
\State Insert new tag at beginning of selection
\end{algorithmic}
\end{algorithm}

Algorithm~\ref{alg:insert_pseudocode} demostrates a simplified procedure to insert a new tag and possibly overwriting the current text selection and thereby mimicking \texttt{execCommand}'s command based on insertion.

\begin{algorithm}
\caption{Simplified text removal pseudocode}
\label{alg:remove_pseudocode}
\begin{algorithmic}[1]
\State If\{Selection is not collapsed\}\{
\State Loop over all nodes inside selection
\State Remove each node
\State Collapse selection\}
\State Else\{
\State Remove one character left of the beginning of the selection\}
\end{algorithmic}
\end{algorithm}

Algorithm~\ref{alg:remove_pseudocode} demonstrates a procedure to mimick the \texttt{delete} command of \texttt{execCommand}.

\begin{algorithm}
\caption{Simplified element unwrapping pseudocode}
\label{alg:unwrap_pseudocode}
\begin{algorithmic}[1]
\State Loop over all nodes inside element
\State Move each node before element
\State Remove element
\end{algorithmic}
\end{algorithm}

Algorithm~\ref{alg:unwrap_pseudocode} demonstrates a procedure to unwrap an element, mimicking the \texttt{removeFormat} and \texttt{unlink} commands of \texttt{execCommand}.
With formatting and removing text as well as inserting and unwrapping elements, we can find equivalents for all commands of the HTML editing APIs related to manipulating rich-text. Aside from this, editing APIs offer commands for undo/redo and clipboard capabilites as well as a command for selecting the entire contents of the editable elements. The latter can easily be implemented with the designated selection API, whereas chapter {ch:implementation} demonstrates an implementation for undo, redo and clipboard capabilities. Table \ref{tab:execCommand_equivalents} lists all commands available for \texttt{execCommand} and possible equivilents with DOM Level 1 methods. However, these are just simplified examples. A specific implementation accounting for all commands and all edge-cases will be discussed in chapters {ch:concept} and {ch:implementation}.

%Insertion commands can be implemented with basic DOM Level 1 methods to create and append elements.

\subsection{Third-party alternatives} The only alternative way to display and edit rich-text inside a browser is through third-party plugins like Adobe Flash or Microsoft Silverlight. Flash and Silverlight lack mobile adoptions and have been subject to critique since the introduction of smartphones and HTML5. Other third-party plugins are even less well adopted. This makes Flash, Silverlight and other third-party browser-plugins a worse choice as compared to displaying and manipulating rich-text though the DOM.