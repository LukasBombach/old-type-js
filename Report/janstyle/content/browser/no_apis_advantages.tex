%!TEX root = ../../thesis.tex

\section{Advantages of rich-text editing without editing APIs}

With a pure JavaScript implementation, many of the problems that editing APIs have, can be solved.

\paragraph{Generated output and flawed API} The generated markup can be chosen with the implementation of the editor. Furthermore, an editor can be implemented to allow developers using the editor to \textit{choose} the output. Section AX describes the inconsistent output across various browsers as well as the problems of the API design of \texttt{execCommand}. Both issues can be adressed by offering a method to wrap the current selection in arbitrary markup. jQuery's \texttt{htmlString} implementation\footnote{\url{http://api.jquery.com/Types/\#htmlString}, last checked on 07/19/2015} demonstrates a simple and stable way to define markup in a string and pass it as an argument to JavaScript methods. A sample call could read as

\begin{lstlisting}[language=JavaScript, caption=Example calls to format text, label=lst:format-examples]
// Mimicking document.execCommand('italic', false, null);
editor.format('<em />');

// Added functionality
editor.format('<span class="highlight" />');
\end{lstlisting}

With an implementation like this, developers using the editor can choose which markup should be generated for italicising text. The markup will be consistent in the scope of their project unless chosen otherwise. Since the DOM manipulation is implemented in JavaScript and not by high-level browser methods, this will also ensure the same output on all systems and solve cross-browser issues. The second example function call in listing \ref{lst:format-examples} demonstrates that custom formatting, fitting the needs of a specific project, can be achieved with the same API.

As described in section AB, many components native to text editing have to be implemented in JavaScript. This requires some effort but also enables full control and direct over it. Ultimately, these components can be exposed in an API to other developers, enabling options for developing editors, not offered by HTML editing APIs. An example API will be discussed in sectionXImplementationn.

\paragraph{Not extendable} When implementing and editor in pure JavaScripts, the limitations aufgedrück by the editing mode, do not apply. Anything that can be implemented in a browser environment can also be implemented as part of a (rich-)text editor. The Google docuement editor demonstrates rich functionality incliding layouting tools or floating images. Both of which are features that are hardly possible in an editing mode enabled environment\footnote{\url{http://googledrive.blogspot.fr/2010/05/whats-different-about-new-google-docs.html}, last checked on 07/19/2015}. SectionABC discusses some use cases exploring the possibilites of rich-text editing implemented this way. % etherpad, markdown etc

\paragraph{Clipboard} In a pure JavaScript environment, clipboard functionality seems to be harder to implement than with the use of editing mode. Apart from filtering the input, pasting is natively available---via keyboard shortcuts as well as the context menu. However, as demonstrated in section IMPLEMENTATION, it is possible to enable native pasting---via keyboard and context menu---even without editing mode. Furthermore, it is possible to filter the pasted contents before inserting them in the editor.

\paragraph{Bugs} No software can be guaranteed to be bug free. However, refraining from using HTML editing APIs will render developing an editor independent from all of these APIs' bugs. Going a step further, the implementation can be aimed to minimize interaction with browser APIs, especially unstable ones. DOM manipulation APIs have been standardized for more than 15 years and tend to be well-proven and stable. This will minimize the number of ''unfixable'' bugs and ultimately free development from being dependent on browser development, update cycles and user adoption. Bugs can be fixed quicker and rolled out to websites. This means that, other than with HTML editing APIs, bugs that occur are part of the library can be fixed and not only worked around. Furthermore, with minimizing browser interaction, bugs probably occur indeopendently of the browser used, which makes finding and fixing bugs easier.


% It puts the contenteditable implementation in the hand of JavaScript developers. We no longer have to wait for browsers to fix issues *and conform each other* and thus can be faster, at least possibly, than browsers are.

% An implementation without editing APIs cannot guarantee to be bug free, but not using these APIS, using only well-proven APIs and minimizing interaction with browser APIs will put the development and the fixing of bugs into the hands of JavaScript delopers. In other words, we \textit{can} actually \textit{fix} bugs and do not have to work around them and wait for browsers and browser usage to change (both takes long time).