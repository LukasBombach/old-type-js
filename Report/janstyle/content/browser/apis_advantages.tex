%!TEX root = ../../thesis.tex

\section{Advantages of HTML Editing APIs}

HTML Editing APIs have some notable advantages which will be discussed in this section.

\paragraph{Browser support}

A fair reason for using HTML editing APIs is its wide browser support. caniuse.com lists browser support for 92.78\% of all used web browsers\cite{caniuse_contenteditable}. I.e. 92.78\% of all people using the web use browsers that have full support for HTML editing APIs.

% \footnote{\url{http://caniuse.com/#search=contenteditable}, last checked on 07/17/2015}

\paragraph{High-level API}

HTML editing APIs offer high-level commands for formatting text. It requires little setup to implement basic rich-text editing. The browser takes care of generating the required markup.

\paragraph{HTML output}

HTML editing APIs modify and generate HTML. In the context of web development, user input in this format is likely to be useful for further processing.

\paragraph{No need for language definitions}

The WHATWG discussed dedicated rich-text inputs, for instance as an extension of the \texttt{textarea} component. Offering a native input for general rich-text input brings up the question which use-cases this input conforms. For a forum software, it might be useful to generate ''BB'' code, while for other purposes other languages might be needed. Offering HTML editing APIs offers a semantically distinct solution, while still enabling a way to implement rich-text editing. %  Conforming the way HTML forms work, offering a native rich-text input, should send the native code that it generates to the server without further processing. That's why it would suck if it sent HTML for BB code editing.

\paragraph{Possbile third-party solutions for other languages}

While HTML editing APIs can be used to generate HTML only, third-party libraries can build on top of that by implementing editors that write ''BB'' code (for instance) and use HTML only for displaying it as rich-text.